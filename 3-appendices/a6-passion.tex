\chapter{Passion}

% Why do people read Finnegan's Wake? It requires a huge amount of dedication and specialized experience to even read a single line of what might ultimately be gibberish. But people respect James Joyce, and he worked on it for 17 years. And many come away from it feeling like they have gleaned something unique and hilarious from this work, a masterpiece crafted in nearly impenetrable language. That's passion. It is the curiosity to understand a complex system of interest.

% B. Ifor Evans writes: "The easiest way to deal with the book would be [...] to write off Mr. Joyce's latest volume as the work of a charlatan. But the author of Dubliners, A Portrait of the Artist and Ulysses is not a charlatan, but an artist of very considerable proportions. I prefer to suspend judgement..."

% So they spend a huge amount of effort to understand it because they think that there is a pot of gold at the end of the rainbow.

% I've read Infinite Jest, which is not nearly as hard, but the experience is likely similar. I earnestly read it because I really enjoyed David Foster Wallace's shorter, more pedestrian works. I certainly didn't catch every detail in the novel, but its style and presentation and content, some aspects of which are purposefully detrimental to its enjoyability, made me think about life in a different way. The use of hundreds of endnotes that force you to constantly flip back and forth is a good example of this. It is physical work to read the book, and the tome becomes weary by the end of it, spine limp and abused.

% The structure under the high-level systems we engage with. The desire to understand some of them in-depth. That's passion.

% I personally need to feel like my work is fulfilling. I cannot just live for the weekends. If I am to give that much time to something, I need to like it. Otherwise, life would be agony to me.
% If you're not passionate about what you're doing, the end product isn't going to be good.
% Intellectual curiosity.

\newpage

