\chapter{How \textit{The Internet} Changed Humanity}

% Maybe do this chronologically from when the first networks were made.
% Maybe do the whole thing in short, punchy sentences written in present tense.
% Maybe throw a timeline on the side (but only after the words are final).

% Mass communication becomes commonplace and pedestrian
% Advertisement become omnipresent
% Privacy dwindles to nothing. Seriously. Unless you have been taking precautionary privacy measures since you started using internet-capable electronics, \textit{they} already know all of your details. The best you can do is stop putting more information out there. It's like throwing coins into a well. You can't get your coins back, but you can stop throwing coins in.
% Illogical ideas spread like wildfire, absurd beliefs find safe havens, headstrong lunatics spawn communities when there normally would not be enough like-minded people in one geographical area for such groups.
% Money-making algorithms stoke the fire of confirmation bias, sending people to sites that further and further solidify people's beliefs. Radicalism booms.
% Social media - a lot to talk about here, the impact on human psychology

% The potential for self-education skyrockets. The Internet becomes the greatest library Man has ever built.
% People under censorship-ridden, authoritarian regimes gain a link to the outside world.
% The world has a chatroom. We can talk to people we will never meet and see perspectives we would never would have seen otherwise. Social barriers disappear in anonymous chatrooms.
% Physical location becomes less important. Jobs can now be "remote," socializing no longer depends so heavily on distance.
% An increase in content consumption and content creation. New forms of content emerge as the resources to build amazing things become commonplace.
% Virtual communities form and develop cultures and histories. Personal time is spent between people who have never met in person. The virtual landscape becomes better designed and becomes remarkably clear. Three-dimensional worlds are built, and there is suddenly an option to live part of our lives in artifical universes. The line between artificial and natural experience blurs.

% The philosophy of futurism. Recognizing the inevitable progress of society and technology. Not resisting it by trying to keep things the way they are. Anticipating contingencies and designing ethical regulations in preparation. Being forward-thinking rather than short-sighted.

\newpage

