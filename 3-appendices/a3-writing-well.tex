\chapter{Writing Well}

% I have been told I have excellent writing skills for pretty much my entire life. In high school, I would regularly get essays back with very few corrections or even nitpicks. I was not graded on writing quality often in college, but when I was, the situation was similar. Many of my family and friends have asked me for writing advice over the years, and I think I give insightful, straightforward critique. However, I sometimes stuggle to understand why writing is so difficult for others. For me, it is just the translation of my stream of consciousness to paper plus some judicious editing. Here, I delve deep into the subconscious choices that I make while writing well.

%----------------------------------------

% What is writing?

% What makes writing good? Illustrative details; a consistent sense of tone; conveying your message in as few words as possible (there are valid uses for being wordy though, i.e. for effect); a logical progression of ideas; making important sentences eye-catching; a vast and precise vocabulary; the restraint to only use words if they fit stylistically (remember: context matters, words have their connotations, and your writing should flow phonetically (or not, if that is your artistic intent)).

% How do you think you can be more efficient (in informative writing)? Here's a good method: estimate the scope of the topic; chunk it up into conceptual sections; find the most fundamental words in each section; research them (technical definitions, casual use, etymology) and branch out to more specific words (raw notes); once you feel like you have a decent grasp of the topic, chunk the raw notes into semantic units and try to find a sequential path through all or most of it (scaffolding); make sure the path you find is the path you actually want (otherwise, you may find yourself straying from your original goal); if it is not the right path, reconsider the purpose of the section; refine the chunked, raw notes into a finer-grained progression of ideas; when the progression is smooth enough, start to express the notes in actual writing on the page, adding style and storytelling elements.

% Don't be afraid to scrap something you worked hard on if it's mediocre. It might feel like you've wasted time, but it was simply a step in the path toward excellence. What matters is that the end product is excellent. The hardest situations are those in which elements of what you've made are excellent---you \textit{really} like the way you handled certain parts---but the larger whole to which they belong simply does not work. It takes a lot of humility to destroy something that you've put a lot of thought and effort into, but it is the mark of a true artist to do so.

% Literal language is best for effective communication, but figurative language has its place. It can make your writing beautiful, make something intentionally vague, or get an idea across in a very few words. However, it should be obvious when you switch between one and the other. Otherwise, you risk making everything you write substanceless.

% Do not shy away from making up words if there are no good alternatives. Your language of choice does not have an elegant word for every idea. But keep this in mind: the best made up words are the ones that are made up of words or roots that already exist.

% Know the writing tools that are at your disposal. Punctuation is key (what is the purpose of a comma, semicolon, colon, em dash? how do they affect the cadence?). You can start sentences with "and" and "but." Paragraph length is up to you, and it will likely vary in effective writing. Use bold and italic styles. Use lists.

% Just because a sentence is really long does not mean it is a "run-on sentence." Sentence fragments may be used for effect.

% The use of italics to indicate vocal stress. Integrating this into writing adds a great detail of information. Suddenly, your writing becomes more like a speech preserved in text.

