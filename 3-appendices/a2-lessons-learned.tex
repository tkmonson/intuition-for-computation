\chapter{Lessons Learned}

% Structure this section in "movements."
% I consider myself to be a Stoic and an Objectivist. I also have great respect for Buddhist thought. Obviously, those words comes with a lot of baggage, and I'm not going to state my position on every nitty-gritty thing associated with those words. This is what I mean when I consider myself a follower of these philosophies...

\vspace{4mm}
\begin{displayquote}
	\textit{The fundamental cause of the trouble is that in the modern world the stupid are cocksure while the intelligent are full of doubt.}
	\begin{flushright}
		---Bertrand Russell
	\end{flushright}
\end{displayquote}
\vspace{4mm}

% Sometimes, what you want is already created and available. Other times, you have to create it yourself. Sometimes, prescribed methods of learning are effective. Other times, you have to learn in your own way to get the knowledge you are after.

% Most authority figures know little more than you do. Most of the advice and opinion you hear is not greatly informed. Place your trust in outside ideas carefully. Get multiple opinions and come to your own conclusion. Otherwise, you'll be as clueless as the majority.
% In particular, be wary of strong opinions that are unsubstantiated, fallacious, or very biased. Being steadfast instead of open to change makes your conclusion stale, not strong.
% People need to realize how hard it is to actually understand something. Maybe then there would be fewer uninformed positions.

% It's all too easy to use a word that is close but ultimately incorrect.
% Technical language is awesome. Much of it is incredibly precise. It makes me want to study linguistics.

% Doing a passion project of this scale changes you.
% In what order did you write this book? Somewhat linearly, but honestly way more spatially scattered than I expected. Taking notes was immensely helpful to the process. Loose words, concept blocks, scaffolding, smoothing...
% A rock-solid understanding of mathematics will allow you to excel in any field you want. Technical fields, languages, making really good art. Math is about patterns and ideas, and everything humans care about are patterns and ideas.
% There is much more freedom and uncertainty in the world than I previously thought. Very little is set in stone.
% Categorization is a useful tool for orienting yourself, but it is possible to go too far down the endless rabbit hole. You cannot seperate reality into neat little piles. I can show you green, and I can show you blue, but there are infinitely many shades in between and we all see them differently. You can measure wavelengths of light all you want, but that's not what color is.
% There are so many distractions today. It is easy to fall into the trap of spending non-negligible amounts of time on mediocre things. We have so little time here before the ride slows to a stop and we must exit to our left. Spend your time on things that are worth a damn. In my case... music, academics, physical fitness, fashion and visual art, writing and language, philosophy. And \textit{good} video games (I explictly qualify video games with \textit{good} because 95% of the games I see are purposeless time-wasters).
% Serious artisans understand and use mathematics. Math takes the guesswork out of things and tells you exactly what you are doing.
% Teaching yourself gives you something that is hard to acquire in a classroom. When a teacher tells you something, you are inclined to just accept it as true (and, more importantly, on the next test) instead of questioning it. Before writing about any topic or even any technical term, I read about it from many different sources until I had come to a personal understanding of the semantic idea behind the letters. Often times, there would be slight inconsistencies between sources, and I would have to read a lot of opinions and comments and articles and etymolgies and histories until I decided that one source was correct and the other was mistaken. Sometimes, I discovered that there were actually multiple correct answers. Self-teaching permits and encourages you to be skeptical and an \textit{independent thinker} in the most literal sense of the phrase: a person whose thought is their own.
% Thoughts about considering the ethics of what you do at work, what you're building.
% Mindset is everything. Your entire reality depends on your mindset.
% Some people just do not see the bigger picture. Some people don't understand abstract art. I don't understand those people. I think it comes from having a narrow perspective, not having an open mind, wanting to always be firmly in the box.
% Embracing the fact that you know very little. Not fooling yourself into thinking that something is obvious.
% What is it all? What is this place with these things and these rules? What does it mean for something to be real? Is anything real?
% It is important to acknowledge how much you actually know about something. It is likely that you vastly overestimate how much you actually understand. And society is such that we \textit{ridicule} lack of understanding---calling people \textit{stupid} is our \textit{modus operandi}, our ever-reliable trump card, the ultimate return that immediately invalidates all points without even considering them because they were written by a stupid person. We should instead be embracing our ignorance. It is a part of us.
% In most reasoned opinions, there is a glimmer of truth. To paint a fuller picture of truth, you need to assimilate and accommodate a lot of viewpoints.
% Being a good communicator is about conveying the thoughts in your mind such that others will have little trouble understanding what you have to say and little opportunity to misinterpret your meaning. Much of this comes down to word choice and understanding pragmatics (context, audience, style).

% It's pointless to think about things absolutely. We can't know anything absolutely. It makes more sense to think about things in terms of models. When you say you believe in something, you are asserting something about reality. If you could hypothetically gather all of the information on what something is and how it behaves (the Truth), your assertion should ideally agree with that information. Therefore, what you say about reality is part of a model of reality. The Universe is indeed a simulation.
% Stereotypes are crude models that roughly describe some members of a group in some ways, but ultimately they are not detailed or nuanced enough to describe any member of the group well. They are, however, easy to use because they are easier to understand than more complex models that require a lot of cognitve accommodation of specialized knowledge.
% Models may be effective in some situations and ineffective in others. Darwinism works for biology, but not for sociology. Progressivism has noble goals, but the model has been used in unethical ways in the past (temperance, eugenics). Context matters. Not all situations are alike.

% Humans are so anthropocentric, and it extends to the personal level: everyone is so egocentric. It is wise to acknowledge not only the bias you have as a human being who is concerned with human issues but also the sources of your intra-human biases: the time period in which you live, the societies and social strata to which you belong, the traits with which you were born, and the experiences you have day to day.
% Statements are only meaningful within certain contexts. Extrapolating anecdotal or weak evidence in order to support beliefs on a large scale is likely to misinform you. Before taking a stance on something, ask yourself if you really know what you're talking about. How much research have you done? Have you read contrasting arguments from multiple sources and come to your own logical conclusion? Are you making assumptions or logical leaps? Are you considering all of the factors? Are you valuing your own direct experience over the actual data collected from the experiences of others? Are you selectively ignoring certain data and aggressively promoting others?

% Society and humanity are so fragile. We think we have advanced so far from what we call ancient times, but really we are still struggling with the same old problems: war, greed, poverty, hunger, pride, inequality, arrogance, ignorance, illogic, tribalism, and \textit{lack of empathy}. Scientific progress has hit warp speed with the invention of the general-purpose computer, ushering in a new era in which we recognize knowledge to be our most precious resource. But technology will not make us wise. Even with these marvelous machines at our disposal, cranking out in some cases \textit{quadrillons} of operations per second, we still cannot figure it out. We are still at each others throats, death-gripping grudges of the past few millennia, ignoring the abject desparation of those suffering in front of us, uninterested in the reams of data billowing over themselves like so many droplets of a breaking wave, consuming and always empty. Computers will not gently usher in peace on earth. Computers will not make us happy, and they will not save us. Computers are tools---it is up to us to choose what we build.

% I typically take longer to do things than my peers, but what I produce is also typically better for it. I cannot thrive in impatient environments that demand slipshod work.

% Greater respect for animal cognition. Greater respect for those that lived long ago. Greater respect for the minds of others at a large scale.

% As we spend trillions of dollars in pursuit of an artificial general intelligence, we simultaneously ignore the millions of natural general intelligences in front of us that are slowly eroding in the whirlpool of poverty. Millions of brains, the most complex and computationally powerful machines known to Man, wasting away in private prisons, homeless, or starving.

% Writing this has made me focus hard on the idea of self-reliance. Going forward, I must not let this distance me from the help offered by others. I must learn to trust people.

% I've gotten really good at knowing how precisely something needs to be done. If it's actual words on paper, I am very meticulous and take my time. If it's notes or ideas or scaffolding, I am now comfortable with jotting something down more sloppily. The key to my awakening was acknowledging my \textit{goal} in writing in great detail paragraphs that I did not intend to publish: to preserve a train of thought. I have not finished thinking about the topic, and I will resume later. Thus, I may not, in this case, actually believe everything I write down. I am undecided, but I don't want to lose my progress. In situations like these, minute attention to things like grammar, word choice, and tone is unnecessary. But when I intend to make some bit of my writing public, I still exert an obsessive level of control over the product. Carefully chosen language really makes a difference.

% Invest in yourself. Your body and mind are all you truly have.

% Boat metaphor for free will from early Christianity. We do not have absolute control of our minds. We do not have control of physical events (input). There is a level of psychological control (algorithm programmability), but it differs for each person. Thus, we do not have complete control over our own mental events. What we seem to have the most control over is how we perceive meaning.

% Approach life like a child: curious, confident, and constantly making mistakes.

% Over time, technologies become more user-friendly because this improves sales. Less intellectual effort is asked of the user. At the same time, more and more of the population grow up with the technology as a part of everyday life. People become less tolerant of mishaps. Slowly but surely, we become uneducated and entitled (and, as a result, easier prey for salesmen).

\newpage

