\chapter{Thoughts on Education}

\vspace{4mm}
\begin{displayquote}
	\textit{Most thought-provoking in our thought-provoking time is that we are still not thinking.}
	\vspace{2mm}
	\begin{flushright}
		---Martin Heidegger
	\end{flushright}
\end{displayquote}
\vspace{4mm}

% http://groupoids.org.uk/context.html (Mathematics in Context)
% "Standardized tests can't measure initiative, creativity, imagination, conceptual thinking, curiosity, effort, irony, judgment, commitment, nuance, good will, ethical reflection, or a host of other valuable dispositions and attributes. What they can measure and count are isolated skills, specific facts and function, content knowledge, the least interesting and least significant aspects of learning." - Bill Ayers

% "Logic studies the Absolute 'in itself'; the philosophy of Nature studies the Absolute 'for itself'; and the philosophy of Spirit studies the Absolute 'in and for itself'." - Frederick Copleston

% "The training of mathematicians has its emphasis on rigour, technique and achievement, and has little emphasis on problem formulation, or concept formulation. By contrast, a study of the history of Mathematics shows that in the applications of Mathematics it is the concepts and language which are often more important than the particular theorems."

%----------------------------------------

% My time at Duke was frustrating. This is my attempt to understand exactly why the experience was such an uphill battle.

% 1. My personal experience with college education at Duke. Critique the university and your own actions while a student there.

% 2. Identify the specific issues. Give alternatives. Discuss what the ideal education would look like.
	% What things need to be studied?
	% In what order should they be studied? Top-down. General to specific.
	% How should they be studied? What is the most efficient way of turning information into knowledge? Focus on meaning and motivation, real-world examples, precise technical language, building a dense foundation of philosophy and math (even if it takes longer), giving everything context in history, etc.

%----------------------------------------

% None of this is to say that my choices played no role in my education. They certainly did. I could have done things differently (though I'd argue that I was, at that age, blind to the alternatives and would have benefited greatly from some guidance). But I put a great deal of effort into learning and felt that I made responsible choices (i.e. choices that would help me excel). Still, I always felt like my knowledge was fragile and skin-deep. I put my faith in the system, and it is only now, after spending some time educating myself outside of the system, that I see clearly the flaws in the pedagogy of my youth.

% Looking back on it, I probably would have preferred a smaller school that offered a more streamlined curriculum and gave me more guidance and support.

% Consider: are you biased in your views toward education because of your own learning difficulties? Some people do very well in school and think education is fine.

% I don't really have a solution for all of this. I'm not even sure how one would go about implementing these changes within an educational structure. It would likely require foundational changes. These are just observations.

%----------------------------------------

% It's always been about the semantics, not the syntax. Experts care about meaning and often solve problems without even putting pencil to paper. The knowledge is ingrained in them, and it is not the symbols that are etched in their brains: it is the abstract objects and relations between them. Why do we teach STEM material the way that we do?

% Teaching syntax over semantics. Anecdote of the dot product. Feeling like things are there for no reason. Math ideas are not invented out of thin air, and they shouldn't be taught that way. They are invented for a meaningful purpose, and we should be teaching the history of their original purpose before generalizing it for use in other applications. This also allows us to understand the original assumptions that were made while constructing the model and thus whether or not our scenario can be properly modeled by it.
% Fuzzy-trace theory in child psychology. People (and especially children) learn better when things are explained qualitatively before they are explained quantitatively.
% Search engines search for things syntactically. It would be great to search for things semantically, but computers currently cannot model natural language semantics. But humans understand semantics well, and we should be teaching semantically instead of putting so much emphasis on memorizing technique.

% What are the liberal arts? Is Duke really a liberal arts school?
% I feel like Duke wasn't really there to teach me anything. It was there to tell me words to look up and to give me a bunch of meaningless work.
% Most of the tests I took were paced way too fast. I understand that professors want to cover everything on the test, but a better solution is to have two-day tests. I did not finish most tests, and I always felt like I could have done significant better if I had just had 30 more minutes. Many of the tests felt like you were not supposed to think while taking them. Instead, you were just supposed to memorize algorithms and apply them as fast as possible. So you memorize everything and forget 95% of it half a year later. It felt like college was not really an environment for learning, but a gauntlet that you go through to prove that you have work-ethic.
% Universities, for the most part, want to teach theoretical material, not applied. They don't want to be trade schools. That's fine. The problem, as I see it, is that they struggle to teach either. There isn't enough philosophical thought to effectively teach theory and there isn't enough curriculum-wide integration of tools to teach application (programming language libraries, software frameworks, training in professional-grade software like those used in CAD).
% Anecdote from mechanical design class. Professional engineer gives a guest lecture and talks about the state of his industry. Tells us that the CAD software we have been trained in is not considered professional-grade and that no one uses it. Tells us that we should be very proficient in CAD, when we've only had about 4 CAD related projects over the course of 3.5 years. In this way, Duke takes the middle path between theoretical and practical in CAD and fails to succeed in either.
% There needs to be more of a focus on history. The best way to understand a concept is to understand the problem that the inventors faced. History also prepares you for the future because it gives you case studies of academic thought (thinking outside of the box, new models).
% Universities are many things to many people. But to undergraduates, they are primarily standards organizations.
% Duke's student culture is obsessed with grades and professionalism rather than with actual intellectualism.
% Culture's influence on education: clubs, Greek life, things that are basically part-time jobs. Being stressed out makes it hard to learn, being depressed makes you stupid and uninterested. Having to choose between studying the amount that you need and subsequently "letting the team down" vs. studying just enough to do all of your homework in order to meet club responsibilities. A culture that promotes always being productive to the point of rigidly organizing recreation is not a culture that effectively fosters learning.
% Homework and tests, the obsession with grades, the dismissal of grades as "not enough" to get the job you want, the 50 and 60 percent test averages, people literally having mental breaks (feeling fractured for over a year after the 350 processor). Elite universities attract the kind of students who are obsessive about grades.
% Why is everyone feverishly taking notes? Taking notes in lecture is *pointless*. The professor should be handing out fully documented notes that cover *all* of the material that is in-scope. You cannot listen to or process what the professor is saying if you are taking notes. Anecdote of machine learning class.
% In order to learn, we need time to do nothing but contemplate.
% Learning should not be such a slog. It should be enjoyable.
% Universities give students too much freedom in class structure and scheduling. I believe that people that young usually do not know what they actually want and need to know, and the classes should reflect that. It should be a more curated, streamlined, interdisciplinary experience for the first two years.

% Why do people dog on art majors? Art is actually a perfectly reasonable thing to study in college. But there is a difference between studying art and providing goods or services. Math is also an art. It just happens to have more useful applications. And that's not "more useful" applications, it's more "useful applications." Successful artists and mathematicians both make useful things that people care about. But learning and applying are different things, and modern college curriculums do not straddle the gap properly.

% Why was the machine learning course so good? 1. Very precise and extensive notes written by the professor were publicly available, 2. homework was not problem sets with 20 problems, they were 2 or 3 problems that were long but very elucidating, 3. many real-world examples were given.
% Why do insights have to be acquired through rote practice? Why can't they just be taught?

% Philosophy and logic should be taught in schools. There should also be more emphasis on learning language formally.

% What if we applied this kind of intuitive learning approach to language acquisition? One-to-one word translations are an artificial construct. Really, words in a foreign language are descriptors of abstract concepts and one-to-one semantic relationships are rare. A more effective approach might be to provide many translations for a single word in order to convey to the learner the abstract concept that is intuitive for native speakers.

% I don't think it is wise to study "soft" subjects without first formulating a "hard" framework that is based in logic. It is putting the cart (the goal) before the horse (the means). Soft subjects involve phenomena that are very complex (i.e. psychology studies the mind, sociology studies systems of millions of actors). The vast majority of students entering university do not have these skills. Students should take the exact same fundamental classes before joining a department and doing applied work. Doing applied work without setting the groundwork is a recipe for narrow and fragile education.
% Part of the problem is advanced credit going into college. Skipping intro courses should not be allowed. You may have seen the material before, but you have not seen it in the way that the university wants you to see it.
% Specifically, I think we need to return to a classical education, including extended formal studies of grammar and logic, Latin and/or Greek, rhetoric, philosophy, and history of mathematics.

% We need to be teaching Stoic principles in schools. Controlling a human mind is really hard, and people need to be taught how to handle their emotions. Before anything else, we experience. We need to teach our children how to experience well.

%---EDUCATION IN PERSONAL COMPUTING------

% Children using GUI computers or computers with a lot of abstraction (Apple, mobile devices). Versus people back in the day using the Commadore 64, where it takes a significant understanding of the device to perform the tasks you want to do.
% I remember, back in my early, early days of programming, hearing people refer to computers as "machines" and thinking that was really weird because I didn't think of computers as machines. I just thought of them as magic boxes that somehow make pretty graphics. I remember printing "Hello World" for the first time and thinking it was stupid because I literally had to type "Hello World" onto the screen to get the computer to show "Hello World" on the screen. It was already on the screen. I just typed it. I was technologically spoiled. I didn't recognize how beautiful those two words really are in the context of the groundbreaking human thought that was required to get there and the implications going forward (AI).
% If you want to produce really excellent work with the aid of a computer, you cannot just assume that computers are black boxes. Elite digital professionals understand how computers work.

\newpage

