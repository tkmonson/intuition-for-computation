\chapter{Bibliography}

\begin{displayquote}
	\textit{The origin of concepts, even for a scholar, is very difficult to trace. For a nonscholar such as me, it is easier. But less accurate.}
	\begin{flushright}
		---Peter Freyd
	\end{flushright}
	\vspace{4mm}
\end{displayquote}

% Honor pledge: I hearby state that I have researched diligently the topics covered in this text and have, to the best of my ability, conveyed my honest understanding of them.
% Philosophy of 98% accuracy.
% Nothing here is plagiarized.
% Copyright.

% Thanks to Wikipedia, Stanford Encyclopedia of Philosophy, math.stackexchange, cs.stackexchange, philosophy.stackexchange, tex.stackexchange, english.stackexchange, Stack Overflow, Encyclopedia of Mathematics, Quora, the Arch Linux wiki. I'm not going to reference every single page I used from these places. I used quite literally thousands of different web pages from these sites.

% Outside of these encyclopedic sources, here are some notable works that I referenced while writing this:

%---PHILOSOPHY OF COMPUTATION------------

% Cognitive Set Theory (Rogers)

% Meditations of First Philosophy (Descartes)
% 'A Brute to the Brutes?': Descartes' Treatment of Animals (Cottingham)
% What Is It Like to Be a Bat? (Nagel)
% Consciousness: Here, There, But Not Everywhere (Koch) [Lecture]
% Quining Qualia (Dennett)
% Consciousness Explained (Dennett)
% Facing Up to the Problem of Consciousness (Chalmers)
% Panpsychism and Panprotopsychism (Chalmers)

% Infants' Metaphysics: The Case of Numerical Identity (Xu, Carey)
% Face perception and processing in early infancy: inborn predispositions and developmental changes (Simion, Di Giorgio)
% Newborns' preference for face-relevent stimuli: Effects of contrast polarity (Farroni, Johnson)
% Self-perception and action in infancy (Rochat)
% The cradle of causal reasoning: newborns' preference for physical causality (Mascalzoni, Regolin)
% Five levels of self-awareness as they unfold early in life (Rochat)

% Timaeus (Plato)
% Phaedrus (Plato)
% The Republic (Plato)
% Physics and Philosophy (Werner Heisenberg)
% Quantum Electrodynamics (Richard Feynman)

% Godel, Escher, Bach (Hofstadter)

%---THEORY OF COMPUTATION----------------

% Godel's Proof (Nagal, Newman)

% Robert C. Martin's "Clean Code"

%---PRACTICAL COMPUTING------------------

% 'I've Got Nothing to Hide' and Other Misunderstandings of Privacy (Solove)

\newpage

