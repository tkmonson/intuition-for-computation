\part*{Philosophy of Computation}
\addcontentsline{toc}{part}{\tocpartglyph Philosophy of Computation}

\vspace{4mm}
\begin{displayquote}
    We're presently in the midst of a third intellectual revolution. The first came with Newton: the planets obey physical laws. The second came with Darwin: biology obeys genetic laws. In today’s third revolution, we're coming to realize that even minds and societies emerge from interacting laws that can be regarded as computations. \textit{Everything is a computation.}
	\vspace{2mm}
	\begin{flushright}
		---Rudy Rucker
	\end{flushright}
\end{displayquote}
\vspace{4mm}

% GP1: True names. Identity. Categorization.

One of the longest and oldest threads that can be found in the tapestry of human culture is the concept of the \textit{true name}. It is mystical in origin, appearing in magical and religious contexts as well as in folklores throughout history, and it is founded on the belief that certain words are imbued with the power to change reality. A true name is an expression of its bearer's true nature or \textit{essence}, a summary of all that they are, and so it is said that to know such a name is to have power over the person or thing to whom or which it refers. Like an arcane spell in a dusty grimoire, it is a precious and potent bit of information in the eyes of those who believe. \\

Of course, magic is not real. That is, magic does not exist in reality, magic is not a special phenomenon that science cannot describe, and magical incantations do not grant casters supernatural abilities. But the curious thing about magic is that its existence depends on belief, and it is, in fact, really and truly real for those who think it is (or it is, at least, as really real as "real" gets). And, to make it all even more complicated, depending on how you define the word 'magic,' the slippery thing has a way of wriggling itself into the rigorous and brow-furrowed discussions of non-believers who dig too deep.

% The true name is a supernatural tool for nullifying supernatural threats.

% Magic might be real... depending on how you define magic.
% Magic and feelings. Axioms, starting points, non-rigorous ideas, pure meaning.
% "Incantation" as slang for a computer command

% GP2: Computer science is not about computers. Or, at least, it is not *necessarily* about computers.

% GP3: Computer science is actually about computation. Ok, then what is computation? A process or event. Pops up in many fields of study (logic, mathematics, linguistics, engineering) and was eventually studied itself (making computer science a very interdisciplinary subject). Performed for the purpose of resolving uncertainty.

% GP4: Confusion is our natural state. By means of our intelligence, we give order to our surroundings.

% GP5: Describe data, information, and knowledge with the help of graphics.

% GP6: Domain / Theory / Model

% GP7: Computation can be modeled like a game. There is a game world, with game pieces of varing properties, with various relations among them. An agent is a game piece with agency (i.e. the ability to execute functions). 

\toclineskip
\section{Cognitive Metaphysics}

\subsection{Things and Universes}

\subsection{Substance and Process}

\subsection{Number and Form}

\subsection{Space, Time, and Causality}

% Numbers?
% Abstraction (representation/intentionality)?
 % Cognitive Metaphysics
\toclineskip
\section{Logic and Language}

 % The Syntax and Semantics of Logic and Language
\toclineskip
\section{Information and Communication}

 % Information and Communication
