%--------------------------------------------------------------------------------
%--------------------------------------------------------------------------------
%    CHAPTER IX: APPENDICIES
%--------------------------------------------------------------------------------
%--------------------------------------------------------------------------------

\part*{Appendices}
\addcontentsline{toc}{part}{\tocpartglyph Appendices}

Summarize the appendices below...

\appendix

%--------------------------------------------------------------------------------
%    APPENDIX: MATHEMATICAL FUNDAMENTALS
%--------------------------------------------------------------------------------

\toclineskip
\section{Mathematical Fundamentals}

% Perhaps this could just be a short list of important mathematics terms. There can be different categories like properties, objects, morphisms, concepts...

% Fundamentals, not foundations. These are foundations in that more advanced content is built on top of them, but that's not why they are included here. If you have *never* taken courses in things like elementary geometry, elementary algebra, and trigonometry, you will probably have trouble grasping the content of this book in first place. This section is not here to teach you these concepts for the first time. It is meant to be a refresher of fundamental concepts, deep ideas that are essential to and ubiquitous in the study of mathematics. This section also serves as a sort of "higher-thought warmup" for those who haven't thought about math since high school. Similar to this book as a whole, this appendix often approaches its material in a philosophical manner, explaining in a straightforward way how to understand concepts that are very abstract.
% What is a/an ... number, set, number set, quantity, space, map, morphism, relation, function, field, algebra, operation, algebraic operation, pair, tuple, combination, permutation, vector, matrix, tensor, shape, structure, change (w/ qualifiers like linear, quadratic, cubic, exponential, logarithmic), order, metric, measure, geometry, topology, infinity, continuity, graph of a function, etc.

\newpage

%--------------------------------------------------------------------------------
%    APPENDIX: SYSTEM DESIGN
%--------------------------------------------------------------------------------

\toclineskip
\section{System Design}

%----------------------------------------

\subsection{Server Technologies}
	\begin{description}
		\item[Domain Name System (DNS) Server] Translates a domain name to an IP address of a server containing the information on the requested website. Could use round-robin, load balancing, or geography to choose a server associated with a certain domain name. An OS or browser can cache DNS results until the result's \textit{time to live} (TTL) expires.
	
		If a local DNS server does not know the IP address of some domain name, it can ask a nearby, larger DNS server if it knows. The biggest DNS servers are called \textit{root servers}, and they are distributed across the world's continents. They return lists of servers that would recognize your requested domain name. These are \textit{authoritative name servers} for the appropriate \textit{top-level domain} (.com, .org, .edu, .ca, .uk, etc.).
		
		There are many kinds of results that can be returned by a DNS server:
		\begin{description}[7mm]
			\item[NS record] Name server. Specifies the names of the DNS servers that can translate a given domain.
			\item[MX record] Mail exchange. Specifies the mail servers that will accept a message.
			\item[A record] Address. Points a name to an IP address.
			\item[CNAME] Canonical. Points a name to another name (google.com to www.google.com) or to an A record.
		\end{description}
	
		\item[Load Balancer] Decides which servers to send requests to based on some criteria (random, round-robin, load, session, Layer 4, Layer 7). Can be implemented in software or hardware. Can generate cookies to send to the user. The user can then send those cookies back to return to the server they were using. Multiple load balancers are needed for horizontal scaled systems.
		
		\item[Reverse Proxy] A web server that acts as an intermediary between clients and backend servers. Forwards requests to the backend and returns their responses. This hides the backend server IPs from the client and allows for a centralized source of information. However, it is a single point of failure, so load balancers are a better choice for horizontally scaled systems.
		
		\item[Application Layer] It may be useful to have a layer of application servers separate from your web servers. This allows you to scale the layers independently. A web server serves content to clients using the HTTP. An application server hosts the logic of the application, which could generate an HTTP file to send to a web server. Web servers are often used as reverse proxies for application servers.
	\end{description}

%----------------------------------------
	
\subsection{Persistent Storage Technologies}
\begin{description}
	\item[Caches] Caching involves putting data that is referenced often on a separate, small memory component. For example, when you stay on the same domain, your OS can cache an IP address so it doesn't have to look up the same IP address every time.
	\begin{description}[7mm]
		\item[File-based caching] The data that you want is saved in a local file. For example, you could cache an HTML file instead of dynamically creating a website with data from a database. Not recommended for scalable solutions
		\item[In-memory caching] Copy the most popular pieces of data from an application's database and put it in RAM for faster access times. If you choose to cache a database query as a key and the result as a value, it is hard to determine when to delete this pair when the data becomes stale. Alternatively, you could cache objects. When an object is instantiated, make the necessary database requests to initialize values, and store the object in memory. If a piece of data changes, delete all objects that use that piece of data from cache. This allows for \textit{asynchronous processing} (the application only touches the database when creating objects). Popular systems include Memcached and Redis.
		\item[Cache Update Strategies] \hspace{1cm}
		\begin{description}[7mm]
			\item[Cache-aside] The cache does not interact with storage directly. The application looks for data in the cache. Upon a cache miss, it finds its data in storage, adds it to the cache, and returns the data. Only requested data is cached. Data in the cache can become stale if it is updated in storage but not in cache.
			\item[Write-through] The application uses the cache as its primary data store, and the cache reads and writes to the database. Application adds data to cache, cache writes to data store and returns value to the application. Writes are synchronous and slow, but data is consistent. Reads of cached data are fast. However, most data written to the cache will not be read.
			\item[Write-back] The application adds data to the cache, and the data is asynchronously written to the database. Data loss could occur if the cache fails before new data is written to the database.
		\end{description}
	\end{description}
	
	\item[RAID] \textit{Redundant Array of Independent Disks} is a technique that uses many physical disk drives to improve the redundancy or performance of a system.
	\begin{description}[7mm]
		\item[RAID0] Writes a portion of a file on one drive and the other portion on another drive concurrently. This doubles write speed but has no redundancy.
		\item[RAID1] Writes the whole file on both drives. No write speed improvements, but improves redundancy.
		\item[RAID10] A combination of RAID0 and RAID1. If you have 4 drives, a file is striped between 2 drives and the same striped data is written concurrently to the other 2 drives.
		\item[RAID5] Given N drives, you stripe data across N-1 drives and stores a full copy of the file on 1 drive.
		\item[RAID6] Given N drives, you stripe data across N-2 drives and stores a full copy of the file on 2 drives.
	\end{description}

	\item[Relational Database Management System (RDBMS)] A \textit{relational database} is a collection of items organized in tables. A \textit{database transaction} is a change in a database. All transactions are ACID: Atomic (all or nothing), Consistent (moves the database from one valid state to another), Isolated (concurrent transactions produce the same results as serial transactions), and Durable (changes do not revert). Various techniques for scaling databases are described below.
	\begin{description}[7mm]
		\item[Replication] \hspace{1cm}
		\begin{description}[7mm]
			\item[Master-slave replication] A master serves reads and writes, replicating writes to the slaves, which can only serve reads. If a master goes offline, the system is read-only until a slave is promoted.
			\item[Master-master replication] There are multiple masters that can serve both reads and writes and coordinate with each other on writes. A master can fail and the system can still be fully-functional. However, writes need to be load balanced. Master-master systems are usually either eventually consistent (not ACID) or have  high-latency, synchronized writes.
			\item[Replication Disadvantages] If a master dies before it can replicate a write, data loss occurs. Lots of write replication to slaves means that slaves cannot serve reads as effectively. More slaves means more replication lag on writes.
		\end{description}
		\item[Federation] Splits up databases by function instead of using a monolithic database. Reads and writes go to the appropriate database, resulting in less replication lag. Smaller databases can fit a greater percentage of results in memory, which allows for more cache hits. Parallel writing is possible between databases. Not effective for large tables.
		\item[Sharding] Data is distributed across different databases (shards) such that each database can only manage a subset of the data (like submitting tests to piles labeled A-M and N-Z). Less traffic, less replication, more cache hits, parallel writes between shards. If one shard goes down, the others can continue to work (however, replication is still necessary to prevent data loss). Load balancing between shards is important. Sharing data between shards could be complicated.
		\item[Denormalization] Improves read performance at the expense of write performance. Redundant copies of data are written in multiple tables to avoid expensive joins.
		\item[SQL Tuning] Benchmark your database system and optimize it by restructuring tables and using appropriate variables.
		
	\end{description}
	
	\item[NoSQL] A NoSQL database stores and retrieves data in ways other than tabular relations. Its transactions are BASE: Basically Available (the system guarantees availability), Soft state (the state of the system may change over time, even without input), and eventually consistent (will become consistent over a period of time, if no further input is received). NoSQL prioritizes availability over consistency. Some configurations are described below.
	\begin{description}[7mm]
		\item[Key-value store] Stores data using keys and values. O(1) reads and writes. Used for simple data model or for rapidly changing data, such as a cache. Complex operations are done in the application layer.
		\item[Document store] All information about an object is stored in a document (XML, JSON, binary, etc.). A document store database provides APIs or a query language to query the documents themselves.
		\item[Wide column store] The basic unit of data is a \textit{column} (name/value pair). Columns can be grouped in column families. Super column families can further group column families. Useful for very large data sets.
		\item[Graph database] Each node is a record and each edge is a relationship between records. Good for many-to-many relationships. Not widely used and relatively new.
	\end{description}
\end{description}

%----------------------------------------

\subsection{Network Techniques}

\begin{description}
	\item[Horizontal Scaling] Distributes your data over many servers. Alleviates the load on a single server, but now requests have to be distributed across these servers. Introduces complexity: load balancers are required, and servers should now be stateless.
	
	\item[Asynchronism] Asynchronous tasks are done to prevent the user from waiting for their results. One example is anticipating user requests and pre-computing their results. Another example is having a worker handle a complicated user job in the background and allowing the user to interact with the application in the meantime. The worker will then signal when the job is complete. A job could also "appear" complete to the user, but require a few additional seconds to actually complete.
	
	\item[Firewalling] A network security system typically used to create a controlled barrier between a trusted internal network and an untrusted external network like the Internet. For example, if you want your server to listen for HTTP and HTTPS requests, you could restrict incoming traffic to only ports 80 and 443. This prevents clients from having, for example, full read and write access to your databases.
	
	\item[Consistency Patterns] When many servers hold copies of the same data, we must find an acceptable method of updating them.
	\begin{description}[7mm]
		\item[Weak consistency] After a write, reads may or may not see it. A best effort approach is taken. Works when the application \textit{must} continue running and data can be lost during outages (VoIP, video chat, multiplayer games). Good for any sort of "live" service.
		\item[Eventual consistency] After a write, read will eventually see it. Works when the application \textit{must} continue running, but data cannot be lost, even during outages (email, blogs, Reddit). Good when writes are important, but reading stale data for a short period of time is acceptable.
		\item[Strong consistency] After a write, reads will see it. Works when everyone needs to see the most up-to-date information at all times, even if it slows the whole system down (file systems, databases). Good when stale data is unacceptable.
	\end{description}

	\item[Shared Session State] If users can access a website from many different servers, how do you keep track of their session data? If a user logs in on one server, how can the network know they are logged in when they move to another server? Store all session data on a single server. Use RAID for redundancy.
	
	\item[Microservices] An application can be structured as a collection of microservices that have their own well-defined independent functions. These microservices can be combined in a modular fashion to create the full application. Each microservice could have its own network architecture. This allows for modular scaling of an application. Software like Apache Zookeeper is used to keep track of microservices and how they interact.
	
	\item[Content Delivery Network (CDN)] A CDN is a globally distributed network of proxy servers that serves content to users from nearby nodes. A \textit{push CDN} only updates some piece of data when the developer pushes data to it. It's faster, but requires more storage on the CDN. A \textit{pull CDN} get content from the developer's server whenever a client requests it. It's slower, but requires less space on the CDN.
\end{description}

\newpage

%--------------------------------------------------------------------------------
%    APPENDIX: LESSONS LEARNED
%--------------------------------------------------------------------------------

\toclineskip
\section{Lessons Learned}

% Sometimes, what you want is already created and available. Other times, you have to create it yourself. Sometimes, prescribed methods of learning are effective. Other times, you have to learn in your own way to get the knowledge you are after.
% Most authority figures know little more than you do. Most of the advice and opinion you hear is not greatly informed. Place your trust in outside ideas carefully. Get multiple opinions and come to your own conclusion. Otherwise, you'll be as clueless as the majority.
% In particular, be wary of strong opinions that are unsubstantiated, fallacious, or very biased. Being steadfast instead of open to change makes your conclusion stale, not strong.
% It's all too easy to use a word that is close but ultimately incorrect.
% Technical language is awesome. Much of it is incredibly precise. It makes me want to study linguistics.
% Doing a passion project of this scale changes you.
% In what order did you write this book? Somewhat linearly, but honestly way more spatially scattered than I expected.
% A rock-solid understanding of mathematics will allow you to excel in any field you want. Technical fields, languages, making really good art. Math is about patterns and ideas, and everything humans care about are patterns and ideas.
% There is much more freedom and uncertainty in the world than I previously thought. Very little is set in stone.
% Categorization is a useful tool for orienting yourself, but it is possible to go too far down the endless rabbit hole. You cannot seperate reality into neat little piles. I can show you green, and I can show you blue, but there are infinitely many shades in between. You can measure wavelengths of light all you want, but that's not what color is.
% There are so many distractions today. It is easy to fall into the trap of spending non-negligible amounts of time on mediocre things. We have so little time here before the ride slows to a stop and we must exit to our left. Spend your time on things that are worth a damn. In my case... music, academics, physical fitness, fashion and visual art, writing and language, philosophy. And \textit{good} video games (I explictly qualify video games with \textit{good} because 95% of the games I see are time-wasters).
% Serious artisans use math. Math takes the guesswork out of things and tells you exactly what you are doing.
% What does it mean for something to be real?
% Teaching yourself gives you something that is harder to discover when you are taught by teachers. When a teacher tells you something, you are inclined to just accept it as true and on the next test instead of questioning it. Before writing about any topic or even any technical word, I read about it from many different sources until I had understood the semantic idea and made my own conclusion about it. Often times, there would be slight inconsistencies between sources, and I would have to read a lot of opinions and comments and articles and etymolgies and histories until I decided that one source was correct and the other was mistaken. Self-teaching permits and encourages you to be skeptical and an \textit{independent thinker} in the most literal sense of the phrase: a person whose thought is their own.
% Thoughts about considering the ethics of what you do at work, what you're building.
% Mindset is everything. Your entire reality depends on your mindset.
% Some people just do not see the bigger picture. Some people don't understand abstract art. I don't understand those people. I think it comes from having a narrow perspective, not having an open mind, wanting to always be firmly in the box.
% Embracing the fact that you know very little. Not fooling yourself into thinking that something is obvious.
% What is it all? What is this place with these things and these rules? Is anything real?
% It is important to acknowledge how much you actually know about something. It is likely that you vastly overestimate how much you actually understand. And society is such that we ridicule lack of understanding---calling people we don't like \textit{stupid} is our modus operandi. We should instead be embracing our ignorance. It is a part of us.
% In most reasoned opinions, there is a glimmer of truth. To paint a full picture of truth, you need to assimilate and accommodate a lot of viewpoints.
% Being a good communicator is about conveying the thoughts in your mind such that others will have little trouble understanding what you have to say and little opportunity to misinterpret your meaning. Much of this comes down to word choice and understanding pragmatics (context, audience, style).

\newpage

%--------------------------------------------------------------------------------
%    APPENDIX: HOW THE INTERNET CHANGED HUMANITY
%--------------------------------------------------------------------------------

\toclineskip
\section{How \textit{The Internet} Changed Humanity}

% Maybe do this chronologically from when the first networks were made.
% Maybe do the whole thing in short, punchy sentences written in present tense.
% Maybe throw a timeline on the side (but only after the words are final).

% Mass communication becomes commonplace and pedestrian
% Advertisement become omnipresent
% Privacy dwindles to nothing. Seriously. Unless you have been taking precautionary privacy measures since you started using internet-capable electronics, \textit{they} already know all of your details. The best you can do is stop putting more information out there. It's like throwing coins into a well. You can't get your coins back, but you can stop throwing coins in.
% Illogical ideas spread like wildfire, absurd beliefs find safe havens, headstrong lunatics spawn communities when there normally would not be enough like-minded people in one geographical area for such groups.
% Money-making algorithms stoke the fire of confirmation bias, sending people to sites that further and further solidify people's beliefs. Radicalism booms.
% Social media - a lot to talk about here, the impact on human psychology

% The potential for self-education skyrockets. The Internet becomes the greatest library Man has ever built.
% People under censorship-ridden, authoritarian regimes gain a link to the outside world.
% The world has a chatroom. We can talk to people we will never meet and see perspectives we would never would have seen otherwise. Social barriers disappear in anonymous chatrooms.
% Physical location becomes less important. Jobs can now be "remote," socializing no longer depends so heavily on distance.
% An increase in content consumption and content creation. New forms of content emerge as the resources to build amazing things become commonplace.

% The philosophy of futurism. Recognizing the inevitable progress of society and technology. Not resisting it by trying to keep things the way they are. Anticipating contingencies and designing ethical regulations in preparation. Being forward-thinking rather than short-sighted.

\newpage

%--------------------------------------------------------------------------------
%    APPENDIX: THOUGHTS ON EDUCATION
%--------------------------------------------------------------------------------

\toclineskip
\section{Thoughts on Education}

% http://groupoids.org.uk/context.html (Mathematics in Context)

% "That is, the training of mathematicians has its emphasis on rigour, technique and achievement, and has little emphasis on problem formulation, or concept formulation. By contrast, a study of the history of Mathematics shows that in the applications of Mathematics it is the concepts and language which are often more important than the particular theorems."

% Scholasticism
% Compare traditional and alternative education.
% Fuzzy-trace theory in child psychology. People (and especially children) learn better when things are explained qualitatively before they are explained quantitatively.

% None of this is to say that my choices played no role in my education. They certainly did. I could have done things differently. But I put a great deal of effort into learning and felt that I made responsible choices (i.e. choices that would help me excel). Still, I always felt like my knowledge was fragile and skin-deep. I put my faith in the system, and it is only now, after spending some time educating myself outside of the system, that I see clearly the flaws in the pedagogy of my youth.

% Looking back on it, I probably would have preferred a smaller school that offered a more streamlined curriculum and gave me more guidance and support.

% Consider: are you biased on your views of education because of your own mind likes to learn? Some people do very well in school and think education is fine.

% I don't really have a solution for all of this. I'm not even sure how one would go about implementing these changes within an educational structure. It would likely require fundamental changes. These are just observations.

% It's always been about the semantics, not the syntax. Experts care about meaning and often solve problems without even putting pencil to paper. The knowledge is ingrained in them, and it is not the symbols that are etched in their brains: it is the abstract objects and relation between them. Why do we teach STEM material the way that we do?

% Teaching syntax over semantics. Anecdote of the dot product. Feeling like things are there for no reason. Math ideas are not invented out of thin air, and they shouldn't be taught that way. They are invented for a meaningful purpose, and we should be teaching the history of their original purpose before generalizing it for use in other applications. This also allows us to understand the original assumptions that were made while constructing the model and thus whether or not our scenario can be properly modeled by it.
% Search engines search for things syntactically. It would be great to search for things semantically, but computers currently cannot model natural language semantics. But humans understand semantics well, and we should be teaching semantically instead of putting so much emphasis on memorizing technique.
% Most of the tests I took were paced way too fast. I understand that professors want to cover everything on the test, but a better solution is to have two-day tests. I did not finish most tests, and I always felt like I could have done significant better if I had just had 30 more minutes. Many of the tests felt like you were not supposed to think while taking them. Instead, you were just supposed to memorize algorithms and apply them as fast as possible. So you memorize everything and forget 95% of it half a year later. It felt like college was not really an environment for learning, but a gauntlet that you go through to prove that you have work-ethic.
% Universities, for the most part, want to teach theoretical material, not applied. They don't want to be trade schools. That's fine. The problem, as I see it, is that they struggle to teach either. There isn't enough philosophical thought to effectively teach theory and there isn't enough curriculum-wide integration of tools to teach application (programming language libraries, frameworks, CAD software).
% There needs to be more of a focus on history. The best way to understand a concept is to understand the problem that the inventors faced. History also prepares you for the future because it gives you case studies of academic thought (thinking outside of the box, new models).
% Culture's influence on education: clubs, Greek life, things that are basically second jobs. Being stressed out makes it hard to learn, being depressed makes you stupid and uninterested. Having to choose between studying the amount that you need and letting "the team" down or studying minimally so you can do all of your homework and meet club responsibilities. A culture that promotes always being productive to the point of rigidly organizing recreation is not a culture that effectively fosters learning.
% Homework and tests, the obsession with grades, the dismissal of grades as "not enough" to get the job you want, the 50 and 60 percent test averages, people literally having mental breaks (feeling fractured for over a year after the 350 processor). Elite universities attract the kind of students who are obsessive about grades.
% Why is everyone feverishly taking notes? Taking notes in lecture is *pointless*. The professor should be handing out fully documented notes that cover *all* of the material that is in-scope. You cannot listen to or process what the professor is saying if you are taking notes. Anecdote of machine learning.
% In order to learn, we need time to do nothing.
% Why do people dog on art majors? Art is actually a perfectly reasonable thing to study in college. But there is a difference between studying art and providing goods or services. Math is also an art. It just happens to have more useful applications. And that's not "more useful" applications, it's more "useful applications." Successful artists and mathematicians both make useful things that people care about. But learning and applying are different things, and modern college curriculums do not straddle the gap properly.
% Why was the machine learning course so good? 1. Very precise and extensive notes written by the professor were publicly available, 2. homework was not problem sets with 20 problems, they were 2 or 3 problems that were long but very elucidating, 3. many real-world examples were given.
% Why do insights have to be acquired through rote practice? Why can't they just be taught?
% Philosophy and logic should be taught in schools. There should also be more emphasis on learning language formally.

% What if we applied this kind of intuitive learning approach to language acquisition? One-to-one word translations are an artificial construct. Really, words in a foreign language are descriptors of abstract concepts and one-to-one semantic relationships are rare. A more effective approach might be to provide many translations for a single word in order to convey to the learner the abstract concept that is intuitive for native speakers.

% I would say there are roughly 3 categories of academic study. In ascending order of truthiness, they are art, science, and "structural" (math, logic, theoretical computer science, anything totally abstract). But all are about patterns.

% "Logic studies the Absolute 'in itself'; the philosophy of Nature studies the Absolute 'for itself'; and the philosophy of Spirit studies the Absolute 'in and for itself'." - Frederick Copleston
% Math, science, art (or, originally, religion)

%---EDUCATION IN PERSONAL COMPUTING------

% Children using GUI computers or computers with a lot of abstraction (Apple, mobile devices). Versus people back in the day using the Commadore 64, where it takes a significant understanding of the device to perform the tasks you want to do.
% I remember, back in my early, early days of programming, hearing people refer to computers as "machines" and thinking that was really weird because I didn't think of computers as machines. I just thought of them as magic boxes that somehow make pretty graphics. I remember printing "Hello World" for the first time and thinking it was stupid because I literally had to type "Hello World" in the code to get the computer to show "Hello World" on the screen. It was already on the screen. I just typed it. I was technologically spoiled. I didn't recognize how beautiful those two words really are in the context of the groundbreaking human thought that was required to get there and the implications going forward (AI).
% If you want to produce really excellent work with the aid of a computer, you cannot just assume that computers are black boxes. Elite digital professionals understand how computers work.

\newpage

%--------------------------------------------------------------------------------
%    APPENDIX: PASSION
%--------------------------------------------------------------------------------

\toclineskip
\section{Passion}

% Why do people read Finnegan's Wake? It requires a huge amount of dedication and specialized experience to even read a single line of what might ultimately be gibberish. But people respect James Joyce, and he worked on it for 17 years. And many come away from it feeling like they have gleaned something unique and hilarious from this work, a masterpiece crafted in nearly impenatrable language. That's passion. It is the curiousity to understand a complex system of interest.

% B. Ifor Evans writes: "The easiest way to deal with the book would be [...] to write off Mr. Joyce's latest volume as the work of a charlatan. But the author of Dubliners, A Portrait of the Artist and Ulysses is not a charlatan, but an artist of very considerable proportions. I prefer to suspend judgement..."

% So they spend a huge amount of effort to understand it because they think that there is a pot of gold at the end of the rainbow.

% I've read Infinite Jest, which is not nearly as hard, but the experience is likely similar. I earnestly read it because I really enjoyed David Foster Wallace's shorter, more pedestrian works. I certainly didn't catch every detail in the novel, but that its style and presentation and content, some aspects of which are purposefully detrimental to its enjoyability, made me think about life in a different way.

% The structure under the high-level systems we engage with. The desire to understand some of them in-depth. That's passion.

% I personally need to feel like my work is fulfilling. I cannot just live for the weekends. If I am to give that much time to something, I need to like it. Otherwise, it will be agonizing to me.
% If you're not passionate about what you're doing, the end product isn't going to be good.
% Intellectual curiousity.

\newpage

%--------------------------------------------------------------------------------
%    APPENDIX: SOURCES, NOT CITATIONS
%--------------------------------------------------------------------------------

\toclineskip
\section{Sources, Not Citations}

\begin{displayquote}
	\textit{The origin of concepts, even for a scholar, is very difficult to trace. For a nonscholar such as me, it is easier. But less accurate.}
	\begin{flushright}
		---Peter Freyd
	\end{flushright}
	\vspace{4mm}
\end{displayquote}

% This guide was not written under the circumstances that a textbook would be. It was not originally intended to be a textbook. However, it is sort of a "narrative textbook" now. I still don't think a bibliography is necessary.
% It was not written with sourcing in mind. I don't consider it to be an "academic work" in the traditional sense of the term. That said, I do consider it a serious work and, as such, the content is rigorously sourced. 
% All other excuses aside, I honestly just didn't care to go through the trouble. And I honestly don't think that most readers would be interested in perusing a dutifully collected bibliography or works cited page that conforms to a style manual even if there was one.
% However, all of the information here is the result of my understanding after doing cross-referenced research.
% Philosophy of 98% accuracy.

% Thanks to Wikipedia, math.stackexchange, cs.stackexchange, philosophy.stackexchange, tex.stackexchange, english.stackexchange, Stack Overflow, Encyclopedia of Mathematics, Stanford Encyclopedia of Philosophy. I'm not going to reference every single page I used from these places. I used quite literally thousands of different web pages from these sites.

% Outside of these encyclopedic sources, here are some notable papers that I referenced while writing this:

% Timaeus
% Phaedrus
% Werner Heisenberg's "Physics and Philosophy"
% Robert C. Martin's "Clean Code"

% Nothing here is plagiarized.
% Copyright.
% If it makes you feel better, you can consider this a work of fiction.

\newpage

%--------------------------------------------------------------------------------
%    APPENDIX: FUTURE WORK
%--------------------------------------------------------------------------------

\toclineskip
\section{Future Work}

% As it stands, I'm proud of what I've built.

% This guide is currently slanted toward computer science. It ventures briefly into the territory of software engineering but does not go far. This is because I lack the required experience to write effectively about software development. I would like to expand this guide to include a more in-depth analysis of practical programming, but I need to spend some time working as a software engineer first.
% I also intend to continue my studies independently. I'm interested in delving deeper into machine learning, artificial intelligence, abstract algebra, modern logic, game theory, control systems, and automata theory. I may end up writing new auxiliary parts on those topics.
% I also have interests in music theory, music critique, fashion, comedy, Stoic philosophy, Buddhism, linguistics and language learning, game design.
