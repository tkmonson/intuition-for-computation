%--------------------------------------------------------------------------------
%--------------------------------------------------------------------------------
%    CHAPTER VII: TOOLS AND TECHNOLOGIES
%--------------------------------------------------------------------------------
%--------------------------------------------------------------------------------

\part*{Tools and Technologies}
\addcontentsline{toc}{part}{\tocpartglyph Tools and Technologies}

"Stupid Computer Shit"

% Application vs. program. Basically the same. Applications are programs that run on an operating system and help the user complete tasks. Some system programs are not applications.

%--------------------------------------------------------------------------------
%    SECTION
%--------------------------------------------------------------------------------

\toclineskip
\section{Operating Systems}

% Multics -> Unix -> BSD -> GNU -> Linux

% Here is what an operating system is. Here is the distribution of modern operating systems. These operating systems are called Unix-like because they are built on top of or heavily influenced by the operating system Unix. Variants of Unix include: BSD, GNU/Linux, macOS, etc. Non Unix-likes include: VAX/VMS, OpenVMS, DOS, OS/2, Microsoft Windows (which is a family of OSs)

% In 1969, AT&T created Unix. It was originally written in assembly, but was rewritten in C by Dennis Ritchie in 1973.
% AT&T lost an antitrust case, preventing them from getting into the computer business. As a result, they had to license Unix to anyone who asked. Unix became very popular as a result because it was a revolutionary OS. Unix is, however, proprietary.
% In 1984, AT&T divested Bell Labs, and Bell Labs started selling Unix.
% Richard Stallman started the GNU project in 1983 to create a free, open-source, Unix compatible OS. The project started with the programs rather than the kernel.
% Linux is basically open-source Unix. It was written from the ground up by Linus Torvalds in 1991. It is based on Minix (mini-Unix), which was an academic, microkernel OS based on Unix. Linux is a monolithic kernel.

% Software law
% Free Software Foundation
% FOSS/FLOSS

% Bare machine - a computer executing instructions without the aid of an operating system

% Real programmers use Linux. Unless they work for Microsoft or Apple. But Linux is demonstrably a better OS for people who are serious about computing.
% Tile-based WMs are better than drag-and-drop WMs. Drag-and-drop allows too much freedom. How often do you need a window that isn't fullscreen, 1/2, or 1/3? Apple is responsible for the "desktop metaphor" and the drag-and-drop trend.

%----------------------------------------

\subsection{Installing Arch Linux}

\subsubsection{A Tour of the Unix File System}

% Unix filesystems: zfs, js, hfx, gps, xfs

% Run tree / -L 1 and run through that list alphabetically, explaining symlinks
% "Binary file" or "binary": an executable program

% /usr originally meant "user" or "users." It may also be interpreted as the backronym "Unix System Resources" or "User System Resources."
% /usr/bin: binaries that come with the OS or binaries that are installed by the OS's package manager
% /usr/local/bin: binaries that are installed locally (by hand, by the user on the machine itself). This is where you should store the scripts that you write.

%----------------------------------------

\subsubsection{Common Commands and Tasks}

% Superuser vs user

% Connecting to WiFi:
% nmcli dev wifi list
% nmcli dev wifi con "SSID" password "password"

\begin{itemize}
	\item chmod +x
	\item .bash\_profile is executed at login for the current user
	\item .bashrc is executed every time a shell is opened for the current user
	\item INI files
	\item Run command (rc) files.
	\item Shell scripts. Shebangs.
	\item Config files (plain text)
	\item Handling swap files in Vim. You accidentally deleted a terminal where you were editing a file, and now you have a file with a previous save and the autosaved file. Which one do you want to look at? You probably want to recover (R) the swap file, save its changes to the main file (:w), reload the file content (:e), and when prompted about the existence of a swap file, delete the swap file (D). If the delete option is not available, that means that the file is being edited elsewhere. Go close those windows.
	\item Daemon - background process
	\item File Descriptors
\end{itemize}

\subsubsection{Making a Personalized Linux Installation}

\begin{itemize}
	\item Install Arch
	\item Download Desktop Environment
	\item Download Window Manager
	\item Download Login Manager (or use startx)
	\item Learn pacman
	\item Download from the AUR
	\item urxvt terminal has its perl extensions and configuration in \textasciitilde/.Xresources (run xrdb \textasciitilde/.Xresources to have the window system grab the changes without a reboot)
	\item urxvt needs monospaced font and fonts that support Unicode
	\item feh
	\item compton
	\item polybar
	\item vim
	\item Change desktop environment on startup by adding exec \textit{ds\_name} to the bottom of \textasciitilde/.xinitrc
\end{itemize}

\toclineskip
\section{Everyday Tools}

%----------------------------------------

\subsection{Essential Programs}

% Pick a good one for each task and stick with it. Learn about the program and get good at it. When you get really comfortable with it, you can consider explore other programs.

% Terminal emulator
% Text editor
% Various IDEs (certain languages are best in a particular IDE)
% Window manager
% Desktop environment
% Torrent client
% Image manipulation app
% Music player app
% Video player app
% What is streaming? What is a streaming client?

\subsection{Version Control}

Blah

\subsubsection{Git}

How do you download stuff from GitHub? There are a few methods that might be available, depending on the software.
\begin{itemize}
	\item git clone the directory
	\item wget raw files
	\item Grab it from a package repo with something like pacman
	\item Get it from the AUR with something like yay or yaourt
	\item Download a tarball, unzip it, extract it, and build the source files into an executable
\end{itemize}

%----------------------------------------

\subsection{Unit Testing}

Blah

\subsubsection{JUnit}

Blah

\subsection{Build Automation}

\subsubsection{Make}

GNU Make compiles source files into executables.

% git clone into /var/git (this is my choice of git folder)
% make; sudo make install

%----------------------------------------

\subsubsection{Maven}

Blah

%----------------------------------------

\subsection{Virtualization and Containerization}

Blah

\subsubsection{Docker}

Blah

%--------------------------------------------------------------------------------
%    SECTION
%--------------------------------------------------------------------------------

\toclineskip
\section{Languages and Language-Likes}

%----------------------------------------

\subsection{Markup and Style}

\subsubsection{TeX}

Blah

\subsubsection{HMTL}

Blah

\subsubsection{CSS}

Blah

%----------------------------------------

\subsection{Data Formats}

Blah

\subsubsection{XML}

Blah

\subsubsection{JSON}

Blah

%----------------------------------------

\subsection{Data Query and Manipulation}

\subsubsection{SQL}

% CRUD

Blah \\
