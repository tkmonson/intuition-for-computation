%--------------------------------------------------------------------------------
%--------------------------------------------------------------------------------
%    CHAPTER VII: TOOLS AND TECHNOLOGIES
%--------------------------------------------------------------------------------
%--------------------------------------------------------------------------------

\part*{Tools and Technologies}
\addcontentsline{toc}{part}{\tocpartglyph Tools and Technologies}

"Stupid Computer Shit"

%--------------------------------------------------------------------------------
%    SECTION
%--------------------------------------------------------------------------------

\toclineskip
\section{Everyday Tools}

%----------------------------------------

\subsection{Essential Programs}

% Text editors, IDEs, window managers etc.

%----------------------------------------

\subsection{Operating Systems}

% What's more "everyday" than an operating system? Here is what an operating system is. Here is the distribution of modern operating systems. These operating systems are called Unix-like because they are built on top of or heavily influenced by the operating system Unix. Variants of Unix include: Linux, BSD, macOS, maybe others. Non Unix-likes include: Microsoft Windows (which is a family of OSs)

% Superuser vs user

\subsubsection{History of Unix and GNU/Linux}

%----------------------------------------

\subsubsection{A Tour of the Unix File System}

% Run tree / -L 1 and run through that list alphabetically, explaining symlinks

%----------------------------------------

\subsubsection{Common Commands and Tasks}

% Store scripts in /usr/local/bin

\begin{itemize}
	\item chmod +x
	\item .bash\_profile is executed at login for the current user
	\item .bashrc is executed every time a shell is opened for the current user
	\item INI files
	\item Run command (rc) files.
	\item Shell scripts. Shebangs.
	\item Config files (plain text)
	\item Handling swap files in Vim. You accidentally deleted a terminal where you were editing a file, and now you have a file with a previous save and the autosaved file. Which one do you want to look at? You probably want to recover (R) the swap file, save its changes to the main file (:w), reload the file content (:e), and when prompted about the existence of a swap file, delete the swap file (D). If the delete option is not available, that means that the file is being edited elsewhere. Go close those windows.
	\item Daemon - background process
	\item File Descriptors
\end{itemize}

\subsubsection{Making a Personalized Linux Installation}

\begin{itemize}
	\item Install Arch
	\item Download Desktop Environment
	\item Download Window Manager
	\item Download Login Manager (or use startx)
	\item Learn pacman
	\item Download from the AUR
	\item urxvt terminal has its perl extensions and configuration in \textasciitilde/.Xresources (run xrdb \textasciitilde/.Xresources to have the window system grab the changes without a reboot)
	\item urxvt needs monospaced font and fonts that support Unicode
	\item feh
	\item compton
	\item polybar
	\item vim
	\item Change desktop environment on startup by adding exec \textit{ds\_name} to the bottom of \textasciitilde/.xinitrc
\end{itemize}

%----------------------------------------

\subsection{Version Control}

Blah

\subsubsection{Git}

How do you download stuff from GitHub? There are a few methods that might be available, depending on the software.
\begin{itemize}
	\item git clone the directory
	\item wget raw files
	\item Grab it from a package repo with something like pacman
	\item Get it from the AUR with something like yay or yaourt
	\item Download a tarball, unzip it, extract it, and build the source files into an executable
\end{itemize}

%----------------------------------------

\subsection{Unit Testing}

Blah

\subsubsection{JUnit}

Blah

\subsection{Build Automation}

\subsubsection{Make}

GNU Make compiles source files into executables.

% git clone into /var/git (this is my choice of git folder)
% make; sudo make install

%----------------------------------------

\subsubsection{Maven}

Blah

%----------------------------------------

\subsection{Virtualization and Containerization}

Blah

\subsubsection{Docker}

Blah

%--------------------------------------------------------------------------------
%    SECTION
%--------------------------------------------------------------------------------

\toclineskip
\section{Languages and Language-Likes}

%----------------------------------------

\subsection{Markup and Style}

\subsubsection{TeX}

Blah

\subsubsection{HMTL}

Blah

\subsubsection{CSS}

Blah

%----------------------------------------

\subsection{Data Formats}

Blah

\subsubsection{XML}

Blah

\subsubsection{JSON}

Blah

%----------------------------------------

\subsection{Data Query and Manipulation}

\subsubsection{SQL}

Blah \\
