%--------------------------------------------------------------------------------
%--------------------------------------------------------------------------------
%    CHAPTER VIII: SOFTWARE DEVELOPMENT AND PHILOSOPHIES
%--------------------------------------------------------------------------------
%--------------------------------------------------------------------------------

% Wiki: continual improvement process, kaizen, genchi genbutsu, The Toyota Way

\part*{Software Development and Philosophies}
\addcontentsline{toc}{part}{\tocpartglyph Software Development and Philosophies}

% Compsci vs software engineering
% Software development vs engineering vs programming
% What is frontend, backend, and full stack?

%--------------------------------------------------------------------------------
%    SECTION
%--------------------------------------------------------------------------------

\toclineskip
\section{Software Engineering Processes and Principles}
% Program life cycle

% Blue box on categories of software engineering questions: pure DS&A problems (anything that sounds mathy), word problems (anything that sounds techy), tool familiarity (knowledge of software principles), tool proficiency (trivia, basically)

\begin{itemize}
	\item Agile
	\item Test Driven Development
	\item "Software development process"
	\item Single Responsibility Principle
	\item Open Closed Principle
\end{itemize}

%--------------------------------------------------------------------------------
%    SECTION
%--------------------------------------------------------------------------------

\toclineskip
\section{Design Patterns}

\begin{itemize}
	\item Singleton
	\item Abstract Factory
\end{itemize}

%----------------------------------------

% Final words (what I want you to take away from all of this):
% The importance of context and judging things relationally.
% The importance of history and of earnestly considering the thoughts of those who came before you.
% Respect for the power of information. Speak carefully.
% There is nothing so complicated that you cannot understand it. If there is, it is something that no one will ever understand. Thus, it is discipline and perseverence that matter, not intelligence.
% You should strive to think logically. Otherwise, you will spread misinformation and skew the models of others.
% Now that you possess the basic tools of creation, you must use them wisely. Do not create without first considering the consequences of doing so. Only build what is good.
% If you are uncertain, take some time to think. The answers lie in logical consistency and a proper evaluation of all of the factors.
% There is an immense amount of information out there today. There is, however, very little knowledge and a dearth of wisdom.
% The overview effect