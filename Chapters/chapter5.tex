%--------------------------------------------------------------------------------
%--------------------------------------------------------------------------------
%    CHAPTER V: PROGRAMMING LANGUAGE THEORY
%--------------------------------------------------------------------------------
%--------------------------------------------------------------------------------

\part*{Programming Language Theory}
\addcontentsline{toc}{part}{\tocpartglyph Programming Language Theory}

% Programming is the act of scheduling something (e.g. cable television programming is the strategic scheduling of different kinds of TV shows). Computer programming is the act of scheduling machine operations, either one by one or in blocks of operations, which are abstracted away and represented by statements written in high-level languages.

% Primitives (derived from primitive notions in philosophy)
% Virtual machine
% REPL
% Entry point
% Interactive programming
% Programming in the large and programming in the small
% Pipeline
% Transcompilation
% Command line interpreter
% Modular programming
% Template processor
% Software framework
% Portability

% Clocking, asynchronous events

% Iteration and recursion are about repeating an operation. How they do that differs.
% Iteration: fast, usually easy to implement
% Recursion: easy to debug, sometimes more natural, uses stack memory
% They can be converted into each other

%--------------------------------------------------------------------------------
%    SECTION
%--------------------------------------------------------------------------------

\toclineskip
\section{Elements of Programming Languages}

% Development: specifications should be separate from implementations

% What does a Turing-complete language need? Conditional branching (not just jumping), others?

% Structured program theorem - any computable function can be represented using three types of control structures (sequence, selection, repetition)

% Statement types: assertion, assignment, goto, return, call

% Field, property, method, object, class, ...

%----------------------------------------

\subsection{Syntax}

%----------------------------------------

\subsection{Type Systems}

%----------------------------------------

\subsection{Control Structures}

%----------------------------------------

\subsection{Libraries}

%----------------------------------------

\subsection{Exceptions}

%----------------------------------------

\subsection{Comments}

%--------------------------------------------------------------------------------
%    SECTION
%--------------------------------------------------------------------------------

\toclineskip
\section{Program Execution}

%--------------------------------------------------------------------------------
%    SECTION
%--------------------------------------------------------------------------------

\toclineskip
\section{A History of Programming Languages}

% What does it mean to print? It means to render a piece of information content in some physical medium. Printers print information into books. Back in the day, when computers didn't have fancy graphical displays, the print command converted digital content (information stored in bits) into a print file or device file, a file that an external device (like a printer) can read and operate on. The operating system of the computer would then output the file to the device, which would render the information content in physical form. Nowadays, the print command converts digital content into various standard file formats and those files output graphically by means of a monitor (and technically, light is physical). Printing hasn't fundamentally changed.
% Turing is imperative and lambda is declarative?
% What is a paradigm? How does having a strict paradigm affect the experience of coding? More restrictive, more streamlined?
% More recent is not necessarily better
% Why are we still using old languages like C?

%--------------------------------------------------------------------------------
%    SECTION
%--------------------------------------------------------------------------------

\toclineskip
\section{Programming Paradigms}

% How can we categorize programming languages?
% Imperative vs. Declarative (Functional and Logic)
	% Declaration-style and expression-style
% Procedural vs. Object-Oriented
% Compiled vs. Interpreted
% Static vs. dynamic
% Weakly typed vs. strongly typed
% Statically typed vs. dynamically typed
% Syntax (sparse, dense, graphical)
% Release date
% Structured vs. Non-structured (Control flow)
% Popular use cases
% Cultural perceptions?
% Company preferences (Microsoft, Apple, general use)
% General-purpose vs domain-specific (MANY domains)
% High-level vs. low-level
% System programming vs. application programming
% Scripting languages (glue, shell, macro, embedded)
% Concurent/Parallel/Distributed/Multithreaded
% Heterogenous programming (OpenCL)
% Natural languages
% Memory management vs. garbage collection (C++ scope-based resource management)
% Performance vs. readability/manageability
% Expressive power
% Whitespace-sensitive or not
% Data languages
% Query languages
% Pointers or not
% Exceptions or not
% Library size
% Numerical computing (e.g. MATLAB, Maple, Mathematica)
% Automated theorem proving
% Hyperlink to esoteric languages lol
% Literate programming (e.g. TeX)
% Ontology language

%----------------------------------------

\subsection{Imperative versus Declarative}

\subsubsection{Functional Programming}

\subsubsection{Logic Programming}

%----------------------------------------

\subsection{Procedural versus Object-Oriented}

%--------------------------------------------------------------------------------
%    SECTION
%--------------------------------------------------------------------------------

\toclineskip
\section{Programming Techniques}

\subsection{Higher-Order Programming}
% Map, filter, fold, zip

\subsubsection{Lambda Expressions}

\subsection{Currying}

\subsection{Metaprogramming}
