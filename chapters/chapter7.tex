%--------------------------------------------------------------------------------
%--------------------------------------------------------------------------------
%    CHAPTER VII: PRACTICAL COMPUTING
%--------------------------------------------------------------------------------
%--------------------------------------------------------------------------------

\part*{Practical Computing}
\addcontentsline{toc}{part}{\tocpartglyph Practical Computing}

"Stupid Computer Shit"

% What is computing? What does it mean to compute?

% Good computer habits are about: controlling what information you receive on a daily basis, performing everyday tasks in the most efficient way possible, being intentional, ...
% I'm going to be honest: most people's computer habits are terrible. I'm going to go out on a limb, dear reader, and say that \textit{your} computer habits are terrible.

% Application vs. program. Basically the same. Applications are programs that run on an operating system and help the user complete tasks. Some system programs are not applications.

% The importance of doing things the hard way. Struggling at first so that you may reach higher heights than those who choose the more accessible tools. The importance of restricting yourself to a few simple tools.

% What is data loss and how can it be prevented?

%--------------------------------------------------------------------------------
%    SECTION
%--------------------------------------------------------------------------------

\toclineskip
\section{Operating Systems}

% Teleprinter, teletype, TTY
% Terminals: hardware terminals and psuedo-terminals (https://unix.stackexchange.com/questions/93531/what-is-stored-in-dev-pts-files-and-can-we-open-them)
% Terminal vs. console (root)

% Multics -> Unix -> BSD -> GNU -> Linux

% Here is what an operating system is. Here is the distribution of modern operating systems. These operating systems are called Unix-like because they are built on top of or heavily influenced by the operating system Unix. Variants of Unix include: BSD, GNU/Linux, macOS, etc. Non Unix-likes include: VAX/VMS, OpenVMS, DOS, OS/2, Microsoft Windows (which is a family of OSs)

% In 1969, AT&T created Unix. It was originally written in assembly, but was rewritten in C by Dennis Ritchie in 1973.
% AT&T lost an antitrust case, preventing them from getting into the computer business. As a result, they had to license Unix to anyone who asked. Unix became very popular as a result because it was a revolutionary OS. Unix is, however, proprietary.
% In 1984, AT&T divested Bell Labs, and Bell Labs started selling Unix.
% Richard Stallman started the GNU project in 1983 to create a free, open-source, Unix compatible OS. The project started with the programs rather than the kernel.
% Linux is basically open-source Unix. It was written from the ground up by Linus Torvalds in 1991. It is based on Minix (mini-Unix), which was an academic, microkernel OS based on Unix. Linux is a monolithic kernel.

% Blue box on privacy
% Software law
% Free Software Foundation
% FOSS/FLOSS and Open Source
% Binary blobs (proprietary device drivers)
% Shoshana Zuboff (The Age of Surveillance Capitalism)

% Computer networks
% The Internet
% Web 1.0, Web 2.0, and Web 3.0 (the Semantic Web)

% Bare machine - a computer executing instructions without the aid of an operating system
% Environment variables ($ indicates that the (all-caps) name following it is a variable; env variables are exported from cmd)
% What is a shell? Why are there so many of them?

% Unix philosophy

% Real programmers use Linux. Unless they work for Microsoft or Apple. But Linux is a demonstrably better OS for people who are serious about computing.
% TUIs (textual/terminal-based user interfaces) / CLIs (command-line interfaces) are better than GUIs (graphical user interfaces). Text is pure content (words). Graphics are appearances (signs). The number of possible signs is much greater than the number of words in any practical language. For practical computing, we want things to be as simple as possible. Graphics should only be used when necessary (browsers, pictures, videos). Symbols can be used in very simple cases (a music player: play, pause, previous track, next track). But computing is more complicated than that and requires a language. Text is the better fit.
% TUIs also prevent repetitive strain injuries and promote home-row typing, which improves the mind-muscle control of your ring and pinkie fingers.
% Corollary to the above: \texttt{vim} is better than \texttt{emacs}.
% Tile-based WMs are better than drag-and-drop (a.k.a. stacking or floating) WMs. Drag-and-drop allows too much freedom. How often do you need a window that isn't fullscreen, 1/2, or 1/3? Apple is responsible for the "desktop metaphor" and the drag-and-drop trend.
% Using Linux via the command-line makes me feel like I am *operating* a computer. It is the right way to compute.

% What is a distribution of Linux? A collection of software based on the Linux kernel that forms an OS.
% Typically includes: Linux kernel, package manager, utility software (such as the GNU utilities), window system, window manager, desktop environment.

%----------------------------------------

\subsection{Installing Arch Linux}

\subsubsection{A Tour of the Unix File System}

% Unix filesystems: zfs, js, hfx, gps, xfs

% Run tree / -L 1 and run through that list alphabetically, explaining symlinks
% "Binary file" or "binary": an executable program

% home is represented with ~ in UNIX because home and ~ shared a keyboard key on the ADM-3A terminal
% /usr originally meant "user" or "users." It may also be interpreted as the backronym "Unix System Resources" or "User System Resources."
% /usr/bin: binaries that come with the OS or binaries that are installed by the OS's package manager
% /usr/local/bin: binaries that are installed locally (by hand, by the user on the machine itself). This is where you should store the scripts that you write.

% Naming conventions for files and directories:
% 1. Case-sensitive
% 2. Upper and lowercase letters, numbers, dot, dash, and underscore allowed (special characters are also technically allowed but must be escaped). \0 (NULL) and / are absolutely forbbiden. 
% 3. File extensions are optional
% 4. Files that start with a dot are hidden.
% 5. Names must be unique inside their directory
% 6. 255-char limit
% Conclusion: the names of files/directories should only use lowercase letters, digits, and either underscore or hyphen (user preference). Other options are just not practical.
% Exceptions are allowed for some files (Java classes start with capitals).
% Spaces are not a good option for file names because spaces separate command arguments. You can use them if you are fond of the aesthetic, but you will have to enquote your files or escape the spaces in their names.
% I prefer kebab case. Just don't start the name with a hyphen because that implies that the name refers to an option.
% my-thing is a file. my-thing/ is a directory.

% Colors can be handled in one of two ways:
% 1. set your background to a single color (or perhaps a stylized background that utilizes a very simple palette of colors) and then download a complementary theme for your terminal
% 2. find a really cool wallpaper with lots of colors and then use pywal to generate a theme for your terminal (which can be hit or miss) 

% Fonts
    % TTF vs. OTF

%----------------------------------------

\subsubsection{Common Commands and Tasks}

% How do commands work? Arguments, options, parameters, subcommands...

% When to write an alias vs. function vs. script?

% Common tasks on a computer:
% Reading and writing
% Downloading and uploading (i.e. writing a remote file to local storage and writing a local file to remote storage)
% Streaming (i.e. reading a remote file)
% Cut/copy/paste (or, in Unix, delete/yank/paste)
% Logging in and signing off
% Start/kill/list processes
% Highlight text (or, in Vim, visual select)
% Navigating a filesystem

% Superuser vs user

% Connecting to WiFi:
% nmcli dev wifi list
% nmcli dev wifi con 'SSID' password 'password'

% Download a file: 
% wget "<address>"
% wget -P <location> "address"

% Kill a process:
% ps aux | grep <process-name>
	% a = show processes for all users
	% u = display the process's user/owner
	% x = also show processes not attached to a terminal
% kill <pid>

% File permissions: read (r), write (w), execute (x)
% 10 chars, positions 0-9
% Position 0: file (-), link (l), directory (d), etc.
% Three triads: user (1-3), group (4-6), and outsider permissions (7-9)

\begin{itemize}
	\item chmod +x
	\item .bash\_profile is executed at login for the current user
	\item .bashrc is executed every time a shell is opened for the current user
	\item INI files
	\item Run command (rc) files.
	\item Shell scripts. Shebangs.
	\item Config files (plain text)
	\item Handling swap files in Vim. You accidentally deleted a terminal where you were editing a file, and now you have a file with a previous save and the autosaved file. Which one do you want to look at? You probably want to recover (R) the swap file, save its changes to the main file (:w), reload the file content (:e), and when prompted about the existence of a swap file, delete the swap file (D). If the delete option is not available, that means that the file is being edited elsewhere. Go close those windows.
	\item Daemon - background process
	\item File Descriptors
\end{itemize}

\subsubsection{Making a Personalized Linux Installation}

% What is ncurses? Important?
% Install necessary fonts (Unicode, Emoji)

\begin{itemize}
	\item Install Arch
	\item Download Desktop Environment
	\item Download Window Manager
	\item Download Login Manager (or use startx)
	\item Learn pacman
	\item Download from the AUR
	\item urxvt terminal has its perl extensions and configuration in \textasciitilde/.Xresources (run xrdb \textasciitilde/.Xresources to have the window system grab the changes without a reboot)
	\item urxvt needs monospaced font and fonts that support Unicode
	\item feh
	\item compton
	\item polybar
	\item vim
	\item Change desktop environment on startup by adding exec \textit{ds\_name} to the bottom of \textasciitilde/.xinitrc
\end{itemize}

\toclineskip
\section{Everyday Tools}

%----------------------------------------

\subsection{Essential Programs}

% Wiki: Utility software

% Pick a good one for each task and stick with it. Learn about the program and get good at it. When you get really comfortable with it, you can consider exploring other programs.

% Shell (bash)
% Terminal emulator (st)
% File manager (ranger)
% Desktop environment (none)
% Window manager (i3)
% Process manager (htop (htop-vim-git), gotop)
% Power management (acpid)
% Screen locking (to get: i3lock or slock)
% Application launcher (dmenu)
% Text editor (vim) (IDEs are not necessary)
% Typesetting system (texlive: includes tex and pdftex programs, latex and context macro packages, and the xetex (a.k.a xelatex) and luatex engines; edit with vimtex)
% Browser (firefox)
% Torrent client (qbittorent)
% Notes (convert markdown files into pdf: pandoc <*.md> -s -o <*.pdf>)
% Image viewer (imv (feh for setting wallpaper))
% Document viewer (zathura)
% Image manipulation app (gimp)
% Music player app (cmus)
% Video player app (vlc)
% Games launcher
% Linter (to get: ale)
% Debugger
% Tester

% What is streaming? What is a streaming client?

% Home is where the user lives. His belongings are usually categorized according to media type. Documents are for text, Downloads are for packages and zips and tarballs, Music is for music, Pictures is for pictures, Videos is for videos.

% What is text? (EOF character)

% Important shell commands and coreutils:
% ls (plus la and lsd)
% ln (create a link)
% cd
% find

% pandoc - a universal document converter

% Vim:
% Geneology: ed -> em -> ex -> vi -> vim (-> neovim)
% If you never learned how to touch-type, vim will teach you.
% :h E<number> (error help)
% Plug-in manager = vim-plug
% vimtex for latex
% Tabs, not spaces! Modify .vimrc to make the tab key insert 4 spaces instead.
% Document tabs in vim: :tabe, :tabc

\subsection{Version Control}

Blah

\subsubsection{Git}

% How do you write a good git commit message?
% It should give context to your changes.
% It should be in the imperative mood.

How do you download stuff from GitHub? There are a few methods that might be available, depending on the software.
\begin{itemize}
	\item git clone the directory
	\item wget raw files
	\item Grab it from a package repo with something like pacman
	\item Get it from the AUR with something like yay or yaourt
	\item Download a tarball, unzip it, extract it, and build the source files into an executable
\end{itemize}

%----------------------------------------

\subsection{Unit Testing}

Blah

\subsubsection{JUnit}

Blah

\subsection{Build Automation}

\subsubsection{Make}

GNU Make compiles source files into executables.

% git clone into /var/git (this is my choice of git folder)
% make; sudo make install

%----------------------------------------

\subsubsection{Maven}

Blah

%----------------------------------------

\subsection{Virtualization and Containerization}

Blah

\subsubsection{Docker}

Blah

%--------------------------------------------------------------------------------
%    SECTION
%--------------------------------------------------------------------------------

\toclineskip
\section{Languages and Language-Likes}

% The Internet: HTML5, CSS3, PHP, JavaScript

%----------------------------------------

\subsection{Markup and Style}

\subsubsection{TeX}

Blah

\subsubsection{HMTL}

Blah

\subsubsection{CSS}

Blah

%----------------------------------------

\subsection{Data Formats}

Blah

\subsubsection{XML}

Blah

\subsubsection{JSON}

Blah

%----------------------------------------

\subsection{Data Query and Manipulation}

\subsubsection{SQL}

% CRUD

Blah \\
