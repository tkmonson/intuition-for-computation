\part*{Philosophy of Computation}
\addcontentsline{toc}{part}{\tocpartglyph Philosophy of Computation}

\vspace{4mm}
\begin{displayquote}
	\textit{We're presently in the midst of a third intellectual revolution. The first came with Newton: the planets obey physical laws. The second came with Darwin: biology obeys genetic laws. In today’s third revolution, we're coming to realize that even minds and societies emerge from interacting laws that can be regarded as computations. \textbf{Everything is a computation.}}
	\vspace{2mm}
	\begin{flushright}
		---Rudy Rucker
	\end{flushright}
\end{displayquote}
\vspace{4mm}

% GP1: Computer science is not about computers. Or, at least, it is not *necessarily* about computers.

% GP2: Computer science is actually about computation. Ok, then what is computation? A process or event. Pops up in many fields of study (logic, mathematics, linguistics, engineering) and was eventually studied itself (making computer science a very interdisciplinary subject). Performed for the purpose of resolving uncertainty.

% GP3: Confusion is our natural state.

% GP4: Describe DIKW with the help of graphics.

% GP5: Domain / Theory / Model

% GP6: Computation can be modeled like a game. There is a game world, with game pieces of varing properties, with various relations among them. An agent is a game piece with agency (i.e. the ability to execute functions). 


\toclineskip
\section{Cognitive Ontology}

\toclineskip
\section{Logic and Epistemology}

\toclineskip
\section{Information and Communication}

