%--------------------------------------------------------------------------------
%--------------------------------------------------------------------------------
%    CHAPTER IX: APPENDICIES
%--------------------------------------------------------------------------------
%--------------------------------------------------------------------------------

\part*{Appendices}
\addcontentsline{toc}{part}{\tocpartglyph Appendices}

Summarize the appendices below...

\appendix

%--------------------------------------------------------------------------------
%    APPENDIX: GLOSSARY
%--------------------------------------------------------------------------------

\toclineskip
\section{Glossary}

%--------------------------------------------------------------------------------
%    APPENDIX: SYSTEM DESIGN
%--------------------------------------------------------------------------------

\toclineskip
\section{System Design}

% This content may actually make it into the main body of the text.

%----------------------------------------

\subsection{Server Technologies}
	\begin{description}
		\item[Domain Name System (DNS) Server] Translates a domain name to an IP address of a server containing the information on the requested website. Could use round-robin, load balancing, or geography to choose a server associated with a certain domain name. An OS or browser can cache DNS results until the result's \textit{time to live} (TTL) expires.
	
		If a local DNS server does not know the IP address of some domain name, it can ask a nearby, larger DNS server if it knows. The biggest DNS servers are called \textit{root servers}, and they are distributed across the world's continents. They return lists of servers that would recognize your requested domain name. These are \textit{authoritative name servers} for the appropriate \textit{top-level domain} (.com, .org, .edu, .ca, .uk, etc.).
		
		There are many kinds of results that can be returned by a DNS server:
		\begin{description}[7mm]
			\item[NS record] Name server. Specifies the names of the DNS servers that can translate a given domain.
			\item[MX record] Mail exchange. Specifies the mail servers that will accept a message.
			\item[A record] Address. Points a name to an IP address.
			\item[CNAME] Canonical. Points a name to another name (google.com to www.google.com) or to an A record.
		\end{description}
	
		\item[Load Balancer] Decides which servers to send requests to based on some criteria (random, round-robin, load, session, Layer 4, Layer 7). Can be implemented in software or hardware. Can generate cookies to send to the user. The user can then send those cookies back to return to the server they were using. Multiple load balancers are needed for horizontal scaled systems.
		
		\item[Reverse Proxy] A web server that acts as an intermediary between clients and backend servers. Forwards requests to the backend and returns their responses. This hides the backend server IPs from the client and allows for a centralized source of information. However, it is a single point of failure, so load balancers are a better choice for horizontally scaled systems.
		
		\item[Application Layer] It may be useful to have a layer of application servers separate from your web servers. This allows you to scale the layers independently. A web server serves content to clients using the HTTP. An application server hosts the logic of the application, which could generate an HTTP file to send to a web server. Web servers are often used as reverse proxies for application servers.
	\end{description}

%----------------------------------------
	
\subsection{Persistent Storage Technologies}
\begin{description}
	\item[Caches] Caching involves putting data that is referenced often on a separate, small memory component. For example, when you stay on the same domain, your OS can cache an IP address so it doesn't have to look up the same IP address every time.
	\begin{description}[7mm]
		\item[File-based caching] The data that you want is saved in a local file. For example, you could cache an HTML file instead of dynamically creating a website with data from a database. Not recommended for scalable solutions
		\item[In-memory caching] Copy the most popular pieces of data from an application's database and put it in RAM for faster access times. If you choose to cache a database query as a key and the result as a value, it is hard to determine when to delete this pair when the data becomes stale. Alternatively, you could cache objects. When an object is instantiated, make the necessary database requests to initialize values, and store the object in memory. If a piece of data changes, delete all objects that use that piece of data from cache. This allows for \textit{asynchronous processing} (the application only touches the database when creating objects). Popular systems include Memcached and Redis.
		\item[Cache Update Strategies] \hspace{1cm}
		\begin{description}[7mm]
			\item[Cache-aside] The cache does not interact with storage directly. The application looks for data in the cache. Upon a cache miss, it finds its data in storage, adds it to the cache, and returns the data. Only requested data is cached. Data in the cache can become stale if it is updated in storage but not in cache.
			\item[Write-through] The application uses the cache as its primary data store, and the cache reads and writes to the database. Application adds data to cache, cache writes to data store and returns value to the application. Writes are synchronous and slow, but data is consistent. Reads of cached data are fast. However, most data written to the cache will not be read.
			\item[Write-back] The application adds data to the cache, and the data is asynchronously written to the database. Data loss could occur if the cache fails before new data is written to the database.
		\end{description}
	\end{description}
	
	\item[RAID] \textit{Redundant Array of Independent Disks} is a technique that uses many physical disk drives to improve the redundancy or performance of a system.
	\begin{description}[7mm]
		\item[RAID0] Writes a portion of a file on one drive and the other portion on another drive concurrently. This doubles write speed but has no redundancy.
		\item[RAID1] Writes the whole file on both drives. No write speed improvements, but improves redundancy.
		\item[RAID10] A combination of RAID0 and RAID1. If you have 4 drives, a file is striped between 2 drives and the same striped data is written concurrently to the other 2 drives.
		\item[RAID5] Given N drives, you stripe data across N-1 drives and stores a full copy of the file on 1 drive.
		\item[RAID6] Given N drives, you stripe data across N-2 drives and stores a full copy of the file on 2 drives.
	\end{description}

	\item[Relational Database Management System (RDBMS)] A \textit{relational database} is a collection of items organized in tables. A \textit{database transaction} is a change in a database. All transactions are ACID: Atomic (all or nothing), Consistent (moves the database from one valid state to another), Isolated (concurrent transactions produce the same results as serial transactions), and Durable (changes do not revert). Various techniques for scaling databases are described below.
	\begin{description}[7mm]
		\item[Replication] \hspace{1cm}
		\begin{description}[7mm]
			\item[Master-slave replication] A master serves reads and writes, replicating writes to the slaves, which can only serve reads. If a master goes offline, the system is read-only until a slave is promoted.
			\item[Master-master replication] There are multiple masters that can serve both reads and writes and coordinate with each other on writes. A master can fail and the system can still be fully-functional. However, writes need to be load balanced. Master-master systems are usually either eventually consistent (not ACID) or have  high-latency, synchronized writes.
			\item[Replication Disadvantages] If a master dies before it can replicate a write, data loss occurs. Lots of write replication to slaves means that slaves cannot serve reads as effectively. More slaves means more replication lag on writes.
		\end{description}
		\item[Federation] Splits up databases by function instead of using a monolithic database. Reads and writes go to the appropriate database, resulting in less replication lag. Smaller databases can fit a greater percentage of results in memory, which allows for more cache hits. Parallel writing is possible between databases. Not effective for large tables.
		\item[Sharding] Data is distributed across different databases (shards) such that each database can only manage a subset of the data (like submitting tests to piles labeled A-M and N-Z). Less traffic, less replication, more cache hits, parallel writes between shards. If one shard goes down, the others can continue to work (however, replication is still necessary to prevent data loss). Load balancing between shards is important. Sharing data between shards could be complicated.
		\item[Denormalization] Improves read performance at the expense of write performance. Redundant copies of data are written in multiple tables to avoid expensive joins.
		\item[SQL Tuning] Benchmark your database system and optimize it by restructuring tables and using appropriate variables.
		
	\end{description}
	
	\item[NoSQL] A NoSQL database stores and retrieves data in ways other than tabular relations. Its transactions are BASE: Basically Available (the system guarantees availability), Soft state (the state of the system may change over time, even without input), and eventually consistent (will become consistent over a period of time, if no further input is received). NoSQL prioritizes availability over consistency. Some configurations are described below.
	\begin{description}[7mm]
		\item[Key-value store] Stores data using keys and values. O(1) reads and writes. Used for simple data model or for rapidly changing data, such as a cache. Complex operations are done in the application layer.
		\item[Document store] All information about an object is stored in a document (XML, JSON, binary, etc.). A document store database provides APIs or a query language to query the documents themselves.
		\item[Wide column store] The basic unit of data is a \textit{column} (name/value pair). Columns can be grouped in column families. Super column families can further group column families. Useful for very large data sets.
		\item[Graph database] Each node is a record and each edge is a relationship between records. Good for many-to-many relationships. Not widely used and relatively new.
	\end{description}
\end{description}

%----------------------------------------

\subsection{Network Techniques}

\begin{description}
	\item[Horizontal Scaling] Distributes your data over many servers. Alleviates the load on a single server, but now requests have to be distributed across these servers. Introduces complexity: load balancers are required, and servers should now be stateless.
	
	\item[Asynchronism] Asynchronous tasks are done to prevent the user from waiting for their results. One example is anticipating user requests and pre-computing their results. Another example is having a worker handle a complicated user job in the background and allowing the user to interact with the application in the meantime. The worker will then signal when the job is complete. A job could also "appear" complete to the user, but require a few additional seconds to actually complete.
	
	\item[Firewalling] A network security system typically used to create a controlled barrier between a trusted internal network and an untrusted external network like the Internet. For example, if you want your server to listen for HTTP and HTTPS requests, you could restrict incoming traffic to only ports 80 and 443. This prevents clients from having, for example, full read and write access to your databases.
	
	\item[Consistency Patterns] When many servers hold copies of the same data, we must find an acceptable method of updating them.
	\begin{description}[7mm]
		\item[Weak consistency] After a write, reads may or may not see it. A best effort approach is taken. Works when the application \textit{must} continue running and data can be lost during outages (VoIP, video chat, multiplayer games). Good for any sort of "live" service.
		\item[Eventual consistency] After a write, read will eventually see it. Works when the application \textit{must} continue running, but data cannot be lost, even during outages (email, blogs, Reddit). Good when writes are important, but reading stale data for a short period of time is acceptable.
		\item[Strong consistency] After a write, reads will see it. Works when everyone needs to see the most up-to-date information at all times, even if it slows the whole system down (file systems, databases). Good when stale data is unacceptable.
	\end{description}

	\item[Shared Session State] If users can access a website from many different servers, how do you keep track of their session data? If a user logs in on one server, how can the network know they are logged in when they move to another server? Store all session data on a single server. Use RAID for redundancy.
	
	\item[Microservices] An application can be structured as a collection of microservices that have their own well-defined independent functions. These microservices can be combined in a modular fashion to create the full application. Each microservice could have its own network architecture. This allows for modular scaling of an application. Software like Apache Zookeeper is used to keep track of microservices and how they interact.
	
	\item[Content Delivery Network (CDN)] A CDN is a globally distributed network of proxy servers that serves content to users from nearby nodes. A \textit{push CDN} only updates some piece of data when the developer pushes data to it. It's faster, but requires more storage on the CDN. A \textit{pull CDN} get content from the developer's server whenever a client requests it. It's slower, but requires less space on the CDN.
\end{description}

\newpage

%--------------------------------------------------------------------------------
%    APPENDIX: LESSONS LEARNED
%--------------------------------------------------------------------------------

\toclineskip
\section{Lessons Learned}

% Structure this section in "movements."
% I consider myself to be a Stoic and an Objectivist. I also have great respect for Buddhist thought. Obviously, those words comes with a lot of baggage, and I'm not going to state my position on every nitty-gritty thing associated with those words. This is what I mean when I consider myself a follower of these philosophies...

\vspace{4mm}
\begin{displayquote}
	\textit{The fundamental cause of the trouble is that in modern world the stupid are cocksure while the intelligent are full of doubt.}
	\begin{flushright}
		---Bertrand Russell
	\end{flushright}
\end{displayquote}
\vspace{4mm}

% Sometimes, what you want is already created and available. Other times, you have to create it yourself. Sometimes, prescribed methods of learning are effective. Other times, you have to learn in your own way to get the knowledge you are after.

% Most authority figures know little more than you do. Most of the advice and opinion you hear is not greatly informed. Place your trust in outside ideas carefully. Get multiple opinions and come to your own conclusion. Otherwise, you'll be as clueless as the majority.
% In particular, be wary of strong opinions that are unsubstantiated, fallacious, or very biased. Being steadfast instead of open to change makes your conclusion stale, not strong.
% People need to realize how hard it is to actually understand something. Maybe then there would be fewer uninformed positions.

% It's all too easy to use a word that is close but ultimately incorrect.
% Technical language is awesome. Much of it is incredibly precise. It makes me want to study linguistics.

% Doing a passion project of this scale changes you.
% In what order did you write this book? Somewhat linearly, but honestly way more spatially scattered than I expected. Taking notes was immensely helpful to the process. Loose words, concept blocks, scaffolding, smoothing...
% A rock-solid understanding of mathematics will allow you to excel in any field you want. Technical fields, languages, making really good art. Math is about patterns and ideas, and everything humans care about are patterns and ideas.
% There is much more freedom and uncertainty in the world than I previously thought. Very little is set in stone.
% Categorization is a useful tool for orienting yourself, but it is possible to go too far down the endless rabbit hole. You cannot seperate reality into neat little piles. I can show you green, and I can show you blue, but there are infinitely many shades in between and we all see them differently. You can measure wavelengths of light all you want, but that's not what color is.
% There are so many distractions today. It is easy to fall into the trap of spending non-negligible amounts of time on mediocre things. We have so little time here before the ride slows to a stop and we must exit to our left. Spend your time on things that are worth a damn. In my case... music, academics, physical fitness, fashion and visual art, writing and language, philosophy. And \textit{good} video games (I explictly qualify video games with \textit{good} because 95% of the games I see are purposeless time-wasters).
% Serious artisans understand and use mathematics. Math takes the guesswork out of things and tells you exactly what you are doing.
% Teaching yourself gives you something that is hard to acquire in a classroom. When a teacher tells you something, you are inclined to just accept it as true (and, more importantly, on the next test) instead of questioning it. Before writing about any topic or even any technical term, I read about it from many different sources until I had come to a personal understanding of the semantic idea behind the letters. Often times, there would be slight inconsistencies between sources, and I would have to read a lot of opinions and comments and articles and etymolgies and histories until I decided that one source was correct and the other was mistaken. Sometimes, I discovered that there were actually multiple correct answers. Self-teaching permits and encourages you to be skeptical and an \textit{independent thinker} in the most literal sense of the phrase: a person whose thought is their own.
% Thoughts about considering the ethics of what you do at work, what you're building.
% Mindset is everything. Your entire reality depends on your mindset.
% Some people just do not see the bigger picture. Some people don't understand abstract art. I don't understand those people. I think it comes from having a narrow perspective, not having an open mind, wanting to always be firmly in the box.
% Embracing the fact that you know very little. Not fooling yourself into thinking that something is obvious.
% What is it all? What is this place with these things and these rules? What does it mean for something to be real? Is anything real?
% It is important to acknowledge how much you actually know about something. It is likely that you vastly overestimate how much you actually understand. And society is such that we \textit{ridicule} lack of understanding---calling people \textit{stupid} is our \textit{modus operandi}, our ever-reliable trump card, the ultimate return that immediately invalidates all points without even considering them because they were written by a stupid person. We should instead be embracing our ignorance. It is a part of us.
% In most reasoned opinions, there is a glimmer of truth. To paint a fuller picture of truth, you need to assimilate and accommodate a lot of viewpoints.
% Being a good communicator is about conveying the thoughts in your mind such that others will have little trouble understanding what you have to say and little opportunity to misinterpret your meaning. Much of this comes down to word choice and understanding pragmatics (context, audience, style).

% It's pointless to think about things absolutely. We can't know anything absolutely. It makes more sense to think about things in terms of models. When you say you believe in something, you are asserting something about reality. If you could hypothetically gather all of the information on what something is and how it behaves (the Truth), your assertion should ideally agree with that information. Therefore, what you say about reality is part of a model of reality. The Universe is indeed a simulation.
% Stereotypes are crude models that roughly describe some members of a group in some ways, but ultimately they are not detailed or nuanced enough to describe any member of the group well. They are, however, easy to use because they are easier to understand than more complex models that require a lot of cognitve accommodation of specialized knowledge.
% Models may be effective in some situations and ineffective in others. Darwinism works for biology, but not for sociology. Progressivism has noble goals, but the model has been used in unethical ways in the past (temperance, eugenics). Context matters. Not all situations are alike.

% Humans are so anthropocentric, and it extends to the personal level: everyone is so egocentric. It is wise to acknowledge not only the bias you have as a human being who is concerned with human issues but also the sources of your intra-human biases: the time period in which you live, the societies and social strata to which you belong, the traits with which you were born, and the experiences you have day to day.
% Statements are only meaningful within certain contexts. Extrapolating anecdotal or weak evidence in order to support beliefs on a large scale is likely to misinform you. Before taking a stance on something, ask yourself if you really know what you're talking about. How much research have you done? Have you read contrasting arguments from multiple sources and come to your own logical conclusion? Are you making assumptions or logical leaps? Are you considering all of the factors? Are you valuing your own direct experience over the actual data collected from the experiences of others? Are you selectively ignoring certain data and aggressively promoting others?

% Society and humanity are so fragile. We think we have advanced so far from what we call ancient times, but really we are still struggling with the same old problems: war, greed, poverty, hunger, pride, inequality, arrogance, ignorance, illogic, tribalism, and \textit{lack of empathy}. Scientific progress has hit warp speed with the invention of the general-purpose computer, ushering in a new era in which we recognize knowledge to be our most precious resource. But technology will not make us wise. Even with these marvelous machines at our disposal, cranking out in some cases \textit{quadrillons} of operations per second, we still cannot figure it out. We are still at each others throats, death-gripping grudges of the past few millennia, ignoring the abject desparation of those suffering in front of us, uninterested in the reams of data billowing over themselves like so many droplets of a breaking wave, consuming and always empty. Computers will not gently usher in peace on earth. Computers will not make us happy, and they will not save us. Computers are tools---it is up to us to choose what we build.

% I typically take longer to do things than my peers, but what I produce is also typically better for it. I cannot thrive in impatient environments that demand slipshod work.

% Greater respect for animal cognition. Greater respect for those that lived long ago. Greater respect for the minds of others at a large scale.

% As we spend trillions of dollars in pursuit of an artificial general intelligence, we simultaneously ignore the millions of natural general intelligences in front of us that are slowly eroding in the whirlpool of poverty. Millions of brains, the most complex and computationally powerful machines known to Man, wasting away in private prisons, homeless, or starving.

% Writing this has made me focus hard on the idea of self-reliance. Going forward, I must not let this distance me from the help offered by others. I must learn to trust people.

% I've gotten really good at knowing how precisely something needs to be done. If it's actual words on paper, I am very meticulous and take my time. If it's notes or ideas or scaffolding, I am now comfortable with jotting something down more sloppily. The key to my awakening was acknowledging my \textit{goal} in writing in great detail paragraphs that I did not intend to publish: to preserve a train of thought. I have not finished thinking about the topic, and I will resume later. Thus, I may not, in this case, actually believe everything I write down. I am undecided, but I don't want to lose my progress. In situations like these, minute attention to things like grammar, word choice, and tone is unnecessary. But when I intend to make some bit of my writing public, I still exert an obsessive level of control over the product. Carefully chosen language really makes a difference.

% Invest in yourself. Your body and mind are all you truly have.

% Boat metaphor for free will from early Christianity. We do not have absolute control of our minds. We do not have control of physical events (input). There is a level of psychological control (algorithm programmability), but it differs for each person. Thus, we do not have complete control over our own mental events. What we seem to have the most control over is how we perceive meaning.

% Approach life like a child: curious, confident, and constantly making mistakes.

% Over time, technologies become more user-friendly because this improves sales. Less intellectual effort is asked of the user. At the same time, more and more of the population grow up with the technology as a part of everyday life. People become less tolerant of mishaps. Slowly but surely, we become uneducated and entitled (and, as a result, easier prey for salesmen).

\newpage

%--------------------------------------------------------------------------------
%    APPENDIX: WRITING WELL
%--------------------------------------------------------------------------------

\toclineskip
\section{Writing Well}

% I have been told I have excellent writing skills for pretty much my entire life. In high school, I would regularly get essays back with very few corrections or even nitpicks. I was not graded on writing quality often in college, but when I was, the situation was similar. Many of my family and friends have asked me for writing advice over the years, and I think I give insightful, straightforward critique. However, I sometimes stuggle to understand why writing is so difficult for others. For me, it is just the translation of my stream of consciousness to paper plus some judicious editing. Here, I delve deep into the subconscious choices that I make while writing well.

%----------------------------------------

% What is writing?

% What makes writing good? Illustrative details; a consistent sense of tone; conveying your message in as few words as possible (there are valid uses for being wordy though, i.e. for effect); a logical progression of ideas; making important sentences eye-catching; a vast and precise vocabulary; the restraint to only use words if they fit stylistically (remember: context matters, words have their connotations, and your writing should flow phonetically (or not, if that is your artistic intent)).

% How do you think you can be more efficient (in informative writing)? Here's a good method: estimate the scope of the topic; chunk it up into conceptual sections; find the most fundamental words in each section; research them (technical definitions, casual use, etymology) and branch out to more specific words (raw notes); once you feel like you have a decent grasp of the topic, chunk the raw notes into semantic units and try to find a sequential path through all or most of it (scaffolding); make sure the path you find is the path you actually want (otherwise, you may find yourself straying from your original goal); if it is not the right path, reconsider the purpose of the section; refine the chunked, raw notes into a finer-grained progression of ideas; when the progression is smooth enough, start to express the notes in actual writing on the page, adding style and storytelling elements.

% Don't be afraid to scrap something you worked hard on if it's mediocre. It might feel like you've wasted time, but it was simply a step in the path toward excellence. What matters is that the end product is excellent. The hardest situations are those in which elements of what you've made are excellent---you \textit{really} like the way you handled certain parts---but the larger whole to which they belong simply does not work. It takes a lot of humility to destroy something that you've put a lot of thought and effort into, but it is the mark of a true artist to do so.

% Literal language is best for effective communication, but figurative language has its place. It can make your writing beautiful, make something intentionally vague, or get an idea across in a very few words. However, it should be obvious when you switch between one and the other. Otherwise, you risk making everything you write substanceless.

% Do not shy away from making up words if there are no good alternatives. Your language of choice does not have an elegant word for every idea. But keep this in mind: the best made up words are the ones that are made up of words or roots that already exist.

% Know the writing tools that are at your disposal. Punctuation is key (what is the purpose of a comma, semicolon, colon, em dash? how do they affect the cadence?). You can start sentences with "and" and "but." Paragraph length is up to you, and it will likely vary in effective writing. Use bold and italic styles. Use lists.

% Just because a sentence is really long does not mean it is a "run-on sentence." Sentence fragments may be used for effect.

% The use of italics to indicate vocal stress. Integrating this into writing adds a great detail of information. Suddenly, your writing becomes more like a speech preserved in text.

%--------------------------------------------------------------------------------
%    APPENDIX: HOW THE INTERNET CHANGED HUMANITY
%--------------------------------------------------------------------------------

\toclineskip
\section{How \textit{The Internet} Changed Humanity}

% Maybe do this chronologically from when the first networks were made.
% Maybe do the whole thing in short, punchy sentences written in present tense.
% Maybe throw a timeline on the side (but only after the words are final).

% Mass communication becomes commonplace and pedestrian
% Advertisement become omnipresent
% Privacy dwindles to nothing. Seriously. Unless you have been taking precautionary privacy measures since you started using internet-capable electronics, \textit{they} already know all of your details. The best you can do is stop putting more information out there. It's like throwing coins into a well. You can't get your coins back, but you can stop throwing coins in.
% Illogical ideas spread like wildfire, absurd beliefs find safe havens, headstrong lunatics spawn communities when there normally would not be enough like-minded people in one geographical area for such groups.
% Money-making algorithms stoke the fire of confirmation bias, sending people to sites that further and further solidify people's beliefs. Radicalism booms.
% Social media - a lot to talk about here, the impact on human psychology

% The potential for self-education skyrockets. The Internet becomes the greatest library Man has ever built.
% People under censorship-ridden, authoritarian regimes gain a link to the outside world.
% The world has a chatroom. We can talk to people we will never meet and see perspectives we would never would have seen otherwise. Social barriers disappear in anonymous chatrooms.
% Physical location becomes less important. Jobs can now be "remote," socializing no longer depends so heavily on distance.
% An increase in content consumption and content creation. New forms of content emerge as the resources to build amazing things become commonplace.
% Virtual communities form and develop cultures and histories. Personal time is spent between people who have never met in person. The virtual landscape becomes better designed and becomes remarkably clear. Three-dimensional worlds are built, and there is suddenly an option to live part of our lives in artifical universes. The line between artificial and natural experience blurs.

% The philosophy of futurism. Recognizing the inevitable progress of society and technology. Not resisting it by trying to keep things the way they are. Anticipating contingencies and designing ethical regulations in preparation. Being forward-thinking rather than short-sighted.

\newpage

%--------------------------------------------------------------------------------
%    APPENDIX: THOUGHTS ON EDUCATION
%--------------------------------------------------------------------------------

\toclineskip
\section{Thoughts on Education}

\vspace{4mm}
\begin{displayquote}
	\textit{Most thought-provoking in our thought-provoking time is that we are still not thinking.}
	\vspace{2mm}
	\begin{flushright}
		---Martin Heidegger
	\end{flushright}
\end{displayquote}
\vspace{4mm}

% http://groupoids.org.uk/context.html (Mathematics in Context)
% "Standardized tests can't measure initiative, creativity, imagination, conceptual thinking, curiosity, effort, irony, judgment, commitment, nuance, good will, ethical reflection, or a host of other valuable dispositions and attributes. What they can measure and count are isolated skills, specific facts and function, content knowledge, the least interesting and least significant aspects of learning." - Bill Ayers

% "Logic studies the Absolute 'in itself'; the philosophy of Nature studies the Absolute 'for itself'; and the philosophy of Spirit studies the Absolute 'in and for itself'." - Frederick Copleston

% "The training of mathematicians has its emphasis on rigour, technique and achievement, and has little emphasis on problem formulation, or concept formulation. By contrast, a study of the history of Mathematics shows that in the applications of Mathematics it is the concepts and language which are often more important than the particular theorems."

%----------------------------------------

% My time at Duke was frustrating. This is my attempt to understand exactly why the experience was such an uphill battle.

% 1. My personal experience with college education at Duke. Critique the university and your own actions while a student there.

% 2. Identify the specific issues. Give alternatives. Discuss what the ideal education would look like.
	% What things need to be studied?
	% In what order should they be studied? Top-down. General to specific.
	% How should they be studied? What is the most efficient way of turning information into knowledge? Focus on meaning and motivation, real-world examples, precise technical language, building a dense foundation of philosophy and math (even if it takes longer), giving everything context in history, etc.

%----------------------------------------

% None of this is to say that my choices played no role in my education. They certainly did. I could have done things differently (though I'd argue that I was, at that age, blind to the alternatives and would have benefited greatly from some guidance). But I put a great deal of effort into learning and felt that I made responsible choices (i.e. choices that would help me excel). Still, I always felt like my knowledge was fragile and skin-deep. I put my faith in the system, and it is only now, after spending some time educating myself outside of the system, that I see clearly the flaws in the pedagogy of my youth.

% Looking back on it, I probably would have preferred a smaller school that offered a more streamlined curriculum and gave me more guidance and support.

% Consider: are you biased in your views toward education because of your own learning difficulties? Some people do very well in school and think education is fine.

% I don't really have a solution for all of this. I'm not even sure how one would go about implementing these changes within an educational structure. It would likely require foundational changes. These are just observations.

%----------------------------------------

% It's always been about the semantics, not the syntax. Experts care about meaning and often solve problems without even putting pencil to paper. The knowledge is ingrained in them, and it is not the symbols that are etched in their brains: it is the abstract objects and relations between them. Why do we teach STEM material the way that we do?

% Teaching syntax over semantics. Anecdote of the dot product. Feeling like things are there for no reason. Math ideas are not invented out of thin air, and they shouldn't be taught that way. They are invented for a meaningful purpose, and we should be teaching the history of their original purpose before generalizing it for use in other applications. This also allows us to understand the original assumptions that were made while constructing the model and thus whether or not our scenario can be properly modeled by it.
% Fuzzy-trace theory in child psychology. People (and especially children) learn better when things are explained qualitatively before they are explained quantitatively.
% Search engines search for things syntactically. It would be great to search for things semantically, but computers currently cannot model natural language semantics. But humans understand semantics well, and we should be teaching semantically instead of putting so much emphasis on memorizing technique.

% What are the liberal arts? Is Duke really a liberal arts school?
% I feel like Duke wasn't really there to teach me anything. It was there to tell me words to look up and to give me a bunch of meaningless work.
% Most of the tests I took were paced way too fast. I understand that professors want to cover everything on the test, but a better solution is to have two-day tests. I did not finish most tests, and I always felt like I could have done significant better if I had just had 30 more minutes. Many of the tests felt like you were not supposed to think while taking them. Instead, you were just supposed to memorize algorithms and apply them as fast as possible. So you memorize everything and forget 95% of it half a year later. It felt like college was not really an environment for learning, but a gauntlet that you go through to prove that you have work-ethic.
% Universities, for the most part, want to teach theoretical material, not applied. They don't want to be trade schools. That's fine. The problem, as I see it, is that they struggle to teach either. There isn't enough philosophical thought to effectively teach theory and there isn't enough curriculum-wide integration of tools to teach application (programming language libraries, software frameworks, training in professional-grade software like those used in CAD).
% Anecdote from mechanical design class. Professional engineer gives a guest lecture and talks about the state of his industry. Tells us that the CAD software we have been trained in is not considered professional-grade and that no one uses it. Tells us that we should be very proficient in CAD, when we've only had about 4 CAD related projects over the course of 3.5 years. In this way, Duke takes the middle path between theoretical and practical in CAD and fails to succeed in either.
% There needs to be more of a focus on history. The best way to understand a concept is to understand the problem that the inventors faced. History also prepares you for the future because it gives you case studies of academic thought (thinking outside of the box, new models).
% Universities are many things to many people. But to undergraduates, they are primarily standards organizations.
% Duke's student culture is obsessed with grades and professionalism rather than with actual intellectualism.
% Culture's influence on education: clubs, Greek life, things that are basically part-time jobs. Being stressed out makes it hard to learn, being depressed makes you stupid and uninterested. Having to choose between studying the amount that you need and subsequently "letting the team down" vs. studying just enough to do all of your homework in order to meet club responsibilities. A culture that promotes always being productive to the point of rigidly organizing recreation is not a culture that effectively fosters learning.
% Homework and tests, the obsession with grades, the dismissal of grades as "not enough" to get the job you want, the 50 and 60 percent test averages, people literally having mental breaks (feeling fractured for over a year after the 350 processor). Elite universities attract the kind of students who are obsessive about grades.
% Why is everyone feverishly taking notes? Taking notes in lecture is *pointless*. The professor should be handing out fully documented notes that cover *all* of the material that is in-scope. You cannot listen to or process what the professor is saying if you are taking notes. Anecdote of machine learning class.
% In order to learn, we need time to do nothing but contemplate.
% Learning should not be such a slog. It should be enjoyable.
% Universities give students too much freedom in class structure and scheduling. I believe that people that young usually do not know what they actually want and need to know, and the classes should reflect that. It should be a more curated, streamlined, interdisciplinary experience for the first two years.

% Why do people dog on art majors? Art is actually a perfectly reasonable thing to study in college. But there is a difference between studying art and providing goods or services. Math is also an art. It just happens to have more useful applications. And that's not "more useful" applications, it's more "useful applications." Successful artists and mathematicians both make useful things that people care about. But learning and applying are different things, and modern college curriculums do not straddle the gap properly.

% Why was the machine learning course so good? 1. Very precise and extensive notes written by the professor were publicly available, 2. homework was not problem sets with 20 problems, they were 2 or 3 problems that were long but very elucidating, 3. many real-world examples were given.
% Why do insights have to be acquired through rote practice? Why can't they just be taught?

% Philosophy and logic should be taught in schools. There should also be more emphasis on learning language formally.

% What if we applied this kind of intuitive learning approach to language acquisition? One-to-one word translations are an artificial construct. Really, words in a foreign language are descriptors of abstract concepts and one-to-one semantic relationships are rare. A more effective approach might be to provide many translations for a single word in order to convey to the learner the abstract concept that is intuitive for native speakers.

% I don't think it is wise to study "soft" subjects without first formulating a "hard" framework that is based in logic. It is putting the cart (the goal) before the horse (the means). Soft subjects involve phenomena that are very complex (i.e. psychology studies the mind, sociology studies systems of millions of actors). The vast majority of students entering university do not have these skills. Students should take the exact same fundamental classes before joining a department and doing applied work. Doing applied work without setting the groundwork is a recipe for narrow and fragile education.
% Part of the problem is advanced credit going into college. Skipping intro courses should not be allowed. You may have seen the material before, but you have not seen it in the way that the university wants you to see it.
% Specifically, I think we need to return to a classical education, including extended formal studies of grammar and logic, Latin and/or Greek, rhetoric, philosophy, and history of mathematics.

% We need to be teaching Stoic principles in schools. Controlling a human mind is really hard, and people need to be taught how to handle their emotions. Before anything else, we experience. We need to teach our children how to experience well.

%---EDUCATION IN PERSONAL COMPUTING------

% Children using GUI computers or computers with a lot of abstraction (Apple, mobile devices). Versus people back in the day using the Commadore 64, where it takes a significant understanding of the device to perform the tasks you want to do.
% I remember, back in my early, early days of programming, hearing people refer to computers as "machines" and thinking that was really weird because I didn't think of computers as machines. I just thought of them as magic boxes that somehow make pretty graphics. I remember printing "Hello World" for the first time and thinking it was stupid because I literally had to type "Hello World" onto the screen to get the computer to show "Hello World" on the screen. It was already on the screen. I just typed it. I was technologically spoiled. I didn't recognize how beautiful those two words really are in the context of the groundbreaking human thought that was required to get there and the implications going forward (AI).
% If you want to produce really excellent work with the aid of a computer, you cannot just assume that computers are black boxes. Elite digital professionals understand how computers work.

\newpage

%--------------------------------------------------------------------------------
%    APPENDIX: PASSION
%--------------------------------------------------------------------------------

\toclineskip
\section{Passion}

% Why do people read Finnegan's Wake? It requires a huge amount of dedication and specialized experience to even read a single line of what might ultimately be gibberish. But people respect James Joyce, and he worked on it for 17 years. And many come away from it feeling like they have gleaned something unique and hilarious from this work, a masterpiece crafted in nearly impenatrable language. That's passion. It is the curiousity to understand a complex system of interest.

% B. Ifor Evans writes: "The easiest way to deal with the book would be [...] to write off Mr. Joyce's latest volume as the work of a charlatan. But the author of Dubliners, A Portrait of the Artist and Ulysses is not a charlatan, but an artist of very considerable proportions. I prefer to suspend judgement..."

% So they spend a huge amount of effort to understand it because they think that there is a pot of gold at the end of the rainbow.

% I've read Infinite Jest, which is not nearly as hard, but the experience is likely similar. I earnestly read it because I really enjoyed David Foster Wallace's shorter, more pedestrian works. I certainly didn't catch every detail in the novel, but that its style and presentation and content, some aspects of which are purposefully detrimental to its enjoyability, made me think about life in a different way. The use of hundreds of endnotes that force you to constantly flip back and forth is a good example of this. It is physical work to read the book, and the tome becomes weary by the end of it, spine limp and abused.

% The structure under the high-level systems we engage with. The desire to understand some of them in-depth. That's passion.

% I personally need to feel like my work is fulfilling. I cannot just live for the weekends. If I am to give that much time to something, I need to like it. Otherwise, life would be agony to me.
% If you're not passionate about what you're doing, the end product isn't going to be good.
% Intellectual curiousity.

\newpage

%--------------------------------------------------------------------------------
%    APPENDIX: BIBLIOGRAPHY
%--------------------------------------------------------------------------------

\toclineskip
\section{Bibliography}

\begin{displayquote}
	\textit{The origin of concepts, even for a scholar, is very difficult to trace. For a nonscholar such as me, it is easier. But less accurate.}
	\begin{flushright}
		---Peter Freyd
	\end{flushright}
	\vspace{4mm}
\end{displayquote}

% Honor pledge: I hearby state that I have researched diligently the topics covered in this text and have, to the best of my ability, conveyed my honest understanding of them.
% Philosophy of 98% accuracy.
% Nothing here is plagiarized.
% Copyright.

% Thanks to Wikipedia, Stanford Encyclopedia of Philosophy, math.stackexchange, cs.stackexchange, philosophy.stackexchange, tex.stackexchange, english.stackexchange, Stack Overflow, Encyclopedia of Mathematics, Quora, the Arch Linux wiki. I'm not going to reference every single page I used from these places. I used quite literally thousands of different web pages from these sites.

% Outside of these encyclopedic sources, here are some notable works that I referenced while writing this:

%---PHILOSOPHY OF COMPUTATION------------

% Cognitive Set Theory (Rogers)

% Meditations of First Philosophy (Descartes)
% 'A Brute to the Brutes?': Descartes' Treatment of Animals (Cottingham)
% What Is It Like to Be a Bat? (Nagel)
% Consciousness: Here, There, But Not Everywhere (Koch) [Lecture]
% Quining Qualia (Dennett)
% Consciousness Explained (Dennett)
% Facing Up to the Problem of Consciousness (Chalmers)
% Panpsychism and Panprotopsychism (Chalmers)

% Infants' Metaphysics: The Case of Numerical Identity (Xu, Carey)
% Face perception and processing in early infancy: inborn predispositions and developmental changes (Simion, Di Giorgio)
% Newborns' preference for face-relevent stimuli: Effects of contrast polarity (Farroni, Johnson)
% Self-perception and action in infancy (Rochat)
% The cradle of causal reasoning: newborns' preference for physical causality (Mascalzoni, Regolin)
% Five levels of self-awareness as they unfold early in life (Rochat)

% Timaeus (Plato)
% Phaedrus (Plato)
% The Republic (Plato)
% Physics and Philosophy (Werner Heisenberg)
% Quantum Electrodynamics (Richard Feynman)

% Godel, Escher, Bach (Hofstadter)

%---THEORY OF COMPUTATION----------------

% Godel's Proof (Nagal, Newman)

% Robert C. Martin's "Clean Code"

%---PRACTICAL COMPUTING------------------

% 'I've Got Nothing to Hide' and Other Misunderstandings of Privacy (Solove)

\newpage

%--------------------------------------------------------------------------------
%    APPENDIX: FUTURE WORK
%--------------------------------------------------------------------------------

\toclineskip
\section{Future Work}

% As it stands, I'm proud of what I've built.

% This guide is currently slanted toward computer science. It ventures briefly into the territory of software engineering but does not go far. This is because I lack the required experience to write effectively about software development. I would like to expand this guide to include a more in-depth analysis of practical programming, but I need to spend some time working as a software engineer first.
% I also intend to continue my studies independently. I'm interested in delving deeper into machine learning, artificial intelligence, abstract algebra, modern logic, game theory, control systems, and automata theory. I may end up writing new auxiliary parts on those topics.
% I also have interests in music theory, music critique, fashion, comedy, Stoic philosophy, Buddhism, linguistics and language learning, game design.
