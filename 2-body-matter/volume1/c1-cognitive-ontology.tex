\chapter{Cognitive Ontology}

\vspace{4mm}
\begin{displayquote}
    \textit{The first rule is to keep an untroubled spirit. The second is to look things in the face and know them for what they are.}
    \vspace{2mm}
    \begin{flushright}
        ---Marcus Aurelius
    \end{flushright}
\end{displayquote}
\vspace{4mm}

Dark turns into light. A boundless zephyr of blurs. You see. You hear. You \textit{feel}. The knife-edge of experience materializes, and you step out onto it. It cuts you, and there is no thought---only sensation. Air kissing skin and turbulent color. Sonic frenzy and the feeling of self-weight. Reality emerges raw and beams through all you are. You are lost in it all: a sailboat in a storm, a satellite adrift in space, and the rest. Indifferent, it rages on. \\

This is roughly the experience of a newborn infant. For children of a month or so, life is a sensory mayhem of little sense. They are not conscious, at least not in the way that we typically think of consciousness. They are, however, aware of their surroundings and sensitive to \textit{phenomena} (i.e. the \textit{appearances} of reality, \textit{observables}). To newborns, the world is not a system of distinct parts but an all-encompassing soup of stimuli. Their sensory organs input data that they do not understand, and they then output \textit{reflexes} honed through millions of years of natural selection. \\

This is not to say that babies are mindless. Rather, their minds are just unlike those of typical adults, which both \textit{filter} and \textit{store} information. For example, while babies and adults have similar hearing ability, babies nevertheless hear things that adults do not. This is because adults subconsciously ignore certain sounds; they may \textit{sense} the auditory data, but they do not \textit{perceive} them. This filtering process is called \textit{sensory gating}, and it inhibits any stimulus that is deemed irrelevant. Any information that is not filtered is then eligible for storage in \textit{memory}. \\

Thus, if the adult mind is like the conical beam of a flashlight, seeing far but neglecting much, the baby mind is like the radiant orb of a lantern, seeing all that is immediate but illuminating nothing deeper. These configurations are well-suited to the goals of each: the adult needs \textit{attention} to understand complex situations whereas the baby needs \textit{unrestricted perception} to acquire as much information as possible. Unburdened by categorical thought and sensory gating, babies live in the present and see the suchness of their surroundings. However, their experience is not quite the \textit{kensh\=o} of Zen---it is more like Aldous Huxley's Mind at Large or William Blake's oft-quoted metaphor, from which Huxley took the title of his first essay on psychonautics: \\

\begin{displayquote}
    If the doors of perception were cleansed every thing would appear to man as it is: Infinite. For man has closed himself up, till he sees all things thro' narrow chinks of his cavern. \\
\end{displayquote}

So it is through a great deal of concurrent filtering and learning that we come to build a cavern around ourselves. But we do not muffle our perception pointlessly---we trade it for genuine \textit{thought}. \\

\section{Upper Ontology}

There is a distinction that is sometimes made in cognitive science between a \textit{(phenomenal) P-consciousness} and an \textit{(access) A-consciousness}, the former being the sole consciousness of the baby mind, concerned only with bits of immediate, subjective experience known as \textit{qualia} (e.g. what it is like to \textit{feel} a delicate raspberry as it touches your tongue, what it is like to \textit{taste} it as you mash it between your molars, the rush of brief and unique \textit{delight} found in enjoying that \textit{particular} berry, etc.) and the latter being the dominant consciousness of the adult mind, concerned with \textit{cognitive information} (e.g. thoughts about sensory data, abstract ideas, memories of the past, plans for the future, and all things involved in \textit{mental computation}). \\

Of course, adults still have a P-consciousness---we can lose ourselves in the moment if we are willing. But ours is a boarded-up P-consciousness, largely unaware of the immense volume of all that is happening at all times. Perhaps it is only the bodhisattvas or the mystics of yore who truly come to know their infant minds again and see the P and A as they are: two \textit{modes} of a unified whole. Or perhaps, more pessimistically (and as Huxley suggests), the P in its unadulterated totality is really a kind of schizophrenic insanity, and it is only by the grace of our specialized neural hardware that we erect defenses against its sensory onslaught. Alas, the suchness of such a consciousness eludes those of us with at least one foot planted solidly on the earth. \\\\

\subsection{The Hard Problem of Consciousness}
What is it about phenomenal consciousness that is so intangible? It has been a question of interest to humanity for thousands of years and an object of formal study since Ren\'e Descartes (1596-1650) first posited the duality of \textit{mind} and \textit{body}. Inspired by the proliferation of \textit{automata} (i.e. self-operating machines) in Paris, Descartes came to suggest the extraordinary: that animals---their limbs, organs, and mannerisms---could be replicated with sufficiently complex machinery. Further, he professed that animals \textit{themselves} were machines, with bone and flesh standing in for wood and metal: \\
        
\begin{displayquote}
    It seems reasonable since art copies nature, and men can make various automata which move without thought, that nature should produce its own automata much more splendid than the artificial ones. These natural automata are the animals. \\
\end{displayquote}
        
Note that Descartes implies here that animals are "without thought." He also claimed that thoughts required a language in which rational ideas could be expressed. (Thoughts encoded in a language, however, need not be expressed outwardly; babies, for example, mentally represent ideas before learning how to render them in speech.) He also reasoned on this basis that an automaton would never be able to think because it would never be able to speak in the way that humans do---by organically producing an appropriate response to any given prompt. (Whether or not he was correct remains to be seen, though the possibility of an \textit{artificial general intelligence} does not seem quite as absurd now as it must have seemed back then.) In Descartes' view, humans are special and cannot be reduced to machinery because, unlike animals, they possess rational minds enriched with a free will that no algorithm can replicate. \\
        
Thus, Descartes championed \textit{substance dualism} (also known as \textit{Cartesian dualism}), in which all things were either fundamentally of \textit{matter (res extensa)} or of immaterial \textit{mind (res cogitans)}. He saw the human being as a union of the two disparate substances but struggled to reconcile his strict dualist views with the decidedly hybrid and experiential character of his own bodily sensations and emotions. In the sixth and final meditation of his \textit{Meditations on First Philosophy}, Descartes states that in being skeptical of his fallible senses, he understands himself to be essentially a mind, a purely thinking thing of no physical extent that is intuitively indivisible. And yet, he nevertheless \textit{feels} the qualia of his body and thus must straddle his own \textit{mind-body duality}, being, in a phrase, both flesh and not: \\
        
\begin{displayquote}
    There is nothing which this nature teaches me more expressly [nor more sensibly] than that I have a body which is adversely affected when I feel pain, which has need of food or drink when I experience the feelings of hunger and thirst, and so on; nor can I doubt there being some truth in all this. \\

    Nature also teaches me by these sensations of pain, hunger, thirst, etc., that I am not only lodged in my body as a pilot in a vessel, but that I am very closely united to it, and so to speak so intermingled with it that I seem to compose with it one whole. For if that were not the case, when my body is hurt, I, who am merely a thinking thing, should not feel pain, for I should perceive this wound by the understanding only, just as the sailor perceives by sight when something is damaged in his vessel; and when my body has need of drink or food, I should clearly understand the fact without being warned of it by confused feelings of hunger and thirst. For all these sensations of hunger, thirst, pain, etc. are in truth none other than certain confused modes of thought which are produced by the union and apparent intermingling of mind and body. \\
\end{displayquote}
        
Fast-forward to the turn of the $20^\textit{th}$ century and general confidence in the scientific method had annealed Descartes' mechanistic philosophy into full-blown metaphysical naturalism. The Modernist movement was in full swing, quantum mechanics was being formulated, mathematics was being axiomatized, and about thirty years later the greatest minds of the era would contribute to a general theory of computation. And in the 1950s, the \textit{cognitive revolution} began, and people started to think of the mind as a complex system that could be explained in terms of information and computation. It seemed like the next logical step in human progress. Like matter had been reduced to elementary particles, like water had been reduced to H$_2$O, like genes had been reduced to DNA, so too would the mind be reduced to its constituent parts. \\
        
And yet, there is also that deep feeling in us that the mind is something else. We are inclined to believe that there is something about consciousness that is incomparable to, say, a Rube Goldberg machine or a computer running a program, no matter how complex either may be. Namely, we feel that there is \textit{something that it is like to experience being}. In 1974, philosopher of mind Thomas Nagel (b. 1937) brought this idea to the attention of the burgeoning field of \textit{cognitive science} with his paper \textit{What Is It Like to Be a Bat?}, in which he states that no physicalist theory will capture the essence of the mind until we come to understand its subjective aspects: \\
        
\begin{displayquote}
    The fact that an organism has conscious experience at all means, basically, that there is something it is like to be that organism. There may be further implications about the form of the experience; there may even (though I doubt it) be implications about the behavior of the organism. But fundamentally an organism has conscious mental states if and only if there is something that it is like to \textit{be} that organism---something it is like \textit{for} the organism. We may call this the subjective character of experience. \\
\end{displayquote}
        
This "subjective character of experience" presents a major challenge to the belief that neuroscience will eventually reduce consciousness to a theory. The \textit{philosophy of mind} is then, perhaps, brazenly Postmodernist---inherently subjective and relativistic, with answers that will remain unknown to mankind in spite of its scientific progress. Nagel posits, for example, that we cannot conceive of the subjective experience of a bat because it is totally alien to our own experience. And further, he argues that it does no good to ground such an experience in greater and greater \textit{objectivity} (i.e. independence from human bias), that such first-person character is stripped away in the third-person framework of science: \\
        
\begin{displayquote}
    It will not help to try to imagine that one has webbing on one's arms, which enables one to fly around at dusk and dawn catching insects in one's mouth; that one has very poor vision, and perceives the surrounding world by a system of reflected high-frequency sound signals; and that one spends the day hanging upside down by one's feet in an attic. In so far as I can imagine this (which is not very far), it tells me only what it would be like for \textit{me} to behave as a bat behaves. But that is not the question. I want to know what it is like for a \textit{bat} to be a bat.
    \begin{center}
        \mydots
    \end{center}
    If the subjective character of experience is fully comprehensible only from one point of view, then any shift to greater objectivity---that is, less attachment to a specific viewpoint---does not take us nearer to the real nature of the phenomenon: it takes us farther away from it. \\
\end{displayquote}
        
For Nagel, this is not necessarily a death knell for any formal understanding of consciousness, but it \textit{is} a declaration of doubt in the capacity of science \textit{as we currently know it} to shed light on such matters. His qualms lie not with physicalism itself---that is, not with the assertion that mental states are fundamentally physical---but with an overconfidence that tends to come with the territory of such a mindset. His goal is to elucidate a problem: that, while it is reasonable to believe that one's mind is the result of solely the neurobiological mechanisms in one's body, it is nevertheless \textit{unreasonable}, if we accept that qualia exist and are essential to our experience, to claim that science is presently capable of capturing the subjective character of the conscious mind within its objective bounds. \\
        
In light of this issue, Nagel concludes with a call for the formulation of an "objective phenomenology," a framework in which qualia could be described independently of their experiencer's point of view. Only then could our subjective character of experience, which is so central to our conception of being human, be expressed "in a form comprehensible to beings incapable of having those experiences." As it stands, we cannot describe our seeing of red without analogy to previous experiences of seeing red, and in such terms we can only convey our meaning to those with eyes and brains like our own. And if bats could talk, they would be similarly unable to communicate their echolocational qualia to our sonar-ignorant minds. Thus, \textit{What Is It Like to Be a Bat?} is not a position piece on the nature of qualia but a declaration of agnosticism toward it: we cannot say anything objective about what we feel until we determine whether or not our feelings have objective content. \\
        
Nevertheless, Nagel is often considered a standard-bearer for \textit{property dualism}, which holds that there are two distinct kinds of properties: \textit{physical} and \textit{mental}. Unlike Cartesian dualism, this position holds that there is only one kind of substance, and, in contemporary Western philosophy, it is almost always considered physical rather than mental, an objective entity rather than a subjective one. But Nagel is particularly a standard-bearer for a variety of property dualism known as \textit{neutral monism}, which offers a middle path: that everything is composed of a neutral stuff that is neither physical nor mental. An early form of this view was put forth by the pragmatic philosopher William James (1842-1910) in his essay \textit{Does Consciousness Exist?}: \\
        
\begin{displayquote}
    My thesis is that if we start with the supposition that there is only one primal stuff or material in the world, a stuff of which everything is composed, and if we call that stuff 'pure experience,' the knowing can easily be explained as a particular sort of relation towards one another into which portions of pure experience may enter. The relation itself is a part of pure experience; one of its 'terms' becomes the subject or bearer of the knowledge, the knower, the other becomes the object known. \\
\end{displayquote}
        
So there is a particular relation $\rightarrow$, one among many, and it is called \textit{knowing}. And it relates one term $A$ to another term $B$. And we say that $A$, the knower, \textit{knows} $B$, the known: $A \rightarrow B$. And all of these---the $\rightarrow$, the $A$, the $B$, and the composed whole---are of a neutral, experiential substance. And, indeed, so it is for everything that we may perceive or conceive. The onus is placed then on the rational interpreter to determine whether a relational structure (or \textit{pattern}) of neutral elements constitutes a physical or mental property. Thus, we are pattern-scanning machines and arbiters of what is \textit{thought} and what is \textit{thing}. \\
        
Regardless of whether or not they exist, qualia are beyond our scope. \\

\subsection{The Categories of the Mind}

\textbf{Space and time} are two features of our reality. As they appear to us, the former is an expanse in three dimensions, and the latter can be thought of as an arrow that extends steadily in one dimension, from past to future. They are inextricably woven into the fabric of our experience, and yet their true nature eludes us. \\

Consider a space that contains many particles of \textit{substance}, all of which move about constantly as time goes on. Suppose that, one by one and randomly, these particles blink out of existence forever. When only one particle remains, is that which is \textit{not} the particle still considered space? Further, can we still say that this particle moves? It remains surrounded by a void, interacting and relating with nothing at all, participating in no distinguishable events. And if indeed nothing changes, does time still tick? If so, will space and time \textit{mean} anything after that last, lonely bit of matter disappears and everything becomes a uniform non-existence? \\

These are the questions that pertain to whether space and time are \textit{absolute} or \textit{relative}. If space is absolute, like a fixed three-dimensional grid that pervades the material realm, we might ask where the origin $(0,0,0)$ is. If such a location were to exist, every object and particle in the Universe would have an absolute position in three coordinates. In other words, position would be a distinguishing feature of each and every object, and it would not be defined in relation to other objects, but to this primordial \textit{center of the Universe}. \\

If space is instead relative, a conceptual origin can be placed wherever one likes. Distances can then be measured from this point. Of course, this can be done in an absolute space as well. Thus, the absolutist viewpoint commits the cardinal sins of being \textit{unobservable} and being \textit{useless}. To any human observer, a reality with absolute space would look no different than a reality with relative space---objects would not change in any way after being related to an arbitrary center. Access to these absolute positions also would not provide us with any additional information---they would just be fixed offsets of any positions measured from a relative point. \\

Perhaps then, space is is better thought of as \textit{relational} rather than \textit{objective}. Perhaps it is not a hidden grid-like object itself, but an \textit{ordering} that is composed of relations between objects. And perhaps then it is also better thought of as a product of the \textit{mind} rather than as a physical thing. That is, space may instead be an \textit{abstraction} that our minds make by estimating the relative distance between two objects and comparing it to every other potential spatial relation in our surroundings. In this view, space is simply a web of relations that is constructed by a sentient being. \\

Similarly, time may not be a hidden clock-like object, but instead a series of relative temporal intervals between \textit{events}. And if we only know that time passes because of the changes we perceive in space, it is possible that time is an abstraction as well, an ordering of events made by a sufficiently conscious mind. Furthermore, if space and time are both relational and coupled, perhaps it is better to think of them as a unified \textit{spacetime} despite our predispositions to think of them separately. \\

The philosophy of space and time is full of eternal questions about infinity, substance, void, uniformity, asymmetry, and consciousness. Comparatively, \textit{computational space and time} are simple. \\

This curiosity aside, we could also design our category system such that quantities and qualities are both considered \textit{properties} of some very general sort of thing. Philosophers have long considered the idea of a highest-level category and have ascribed a variety of names to its members: \textit{entity}, \textit{being}, \textit{thing}, \textit{term}, \textit{individual}, etc. Each carries its own subtle connotations. We will use the name \textit{object} because it is the default choice in current philosophical practice and because it has bled into the technical language of computer science with its meaning largely unchanged. \\

How general is this category of objects? Namely, are properties objects? If they are, then our system collapses to a single category, which really means that no distinctions about reality can be made at all. The category of objects thus becomes the soupy mess we are trying to escape. For now, let us assume that properties are \textit{not} objects. Instead, they are something else, and they \textit{relate} to objects in some way, giving them their characteristics. From these relations, the diversity of existence unfolds. \\

The \textbf{abstract-concrete dichotomy} classifies \textit{things} as being either \textit{abstract} or \textit{concrete}. Despite the fact that most people recognize a difference between an abstract and a concrete thing, this dichotomy does not have a universally accepted definition. That said, being abstract is often defined as being both \textit{non-spatiotemporal} and \textit{causally inefficacious}. That is, something is abstract if and only if 

\begin{enumerate}
    \item it does not exist in \textit{space} or in \textit{time} (or, alternatively, in \textit{spacetime}), and
    \item it cannot be the \textit{cause} of any \textit{effect}.
\end{enumerate}

However, one might argue that nothing is truly causally inefficacious. Indeed, if something was truly unable to cause \textit{anything}, it would not be able to affect the \textit{thoughts} of any philosopher, mathematician, or scientist for whom such an abstract thing would be of interest. And yet \textit{mathematical objects} (e.g. a number, a line, a cube, the sine function, etc.) are often considered abstract, despite the fact that they affect the thoughts of mathematicians daily, thoughts which in turn prompt concrete behavior, such as drawing diagrams. Perhaps, causal inefficacy is too strict a requirement. \\

Let us consider why one might want to refer to something that is abstract. An abstract object references \textit{properties} that many concrete, causally efficacious objects have. For example, a sphere does not exist in space or time; it is \textit{nowhere} and \textit{never}. It also does not seem capable of causing an effect in the way that, say, a ping-pong ball can. And while the abstract sphere may not have the same sort of causal efficacy as concrete spherical objects, it does have a \textit{relation} to those objects. That is, the concept of sphericality is \textit{related} to the \textit{class} of concrete objects that may be reasonably modeled as spheres. Thus, an abstract object is, at least, convenient because it can identify a class of concrete objects that is of interest by specifying relevant properties. \\

In light of recent advancements in fields like neuroscience and artificial intelligence, many believe that the human mind is nothing more than a physical system, albeit a particularly complicated one, that can be modeled mathematically like any other. So it may be that abstract objects are just interpretations of concrete electrochemical signals in our brains. Regardless,  \\

A common approach to this challenge is creating a system of \textit{categories} that partition reality at the highest level. For example, one could suggest that there is a fundamental difference between a \textit{quantity} and a \textit{quality}. If these are considered distinct categories, we can classify \textit{measurable} or \textit{countable} stuff like \textit{length}, \textit{mass}, \textit{number}, and \textit{monetary value} as quantitative and \textit{experiential} stuff like \textit{softness}, \textit{roundness}, \textit{flavor}, and \textit{beauty} as qualitative. \\

The quantity-quality distinction is a generally accepted one. However, the dichotomy is not so absolute. For example, softness is listed above as a quality. We might expect, then, that hardness is a quality. For example, rocks are hard and puppy ears are soft. But hardness is also a quantitative measurement of resistance to plastic deformation. Thus, the word \textit{hardness} can refer to both quantitative and qualitative phenomena. The categorical difference between these hardnesses, which exists independent of language, is identified through the context in which the word is used. A hardness can be measured with numbers, but it can also simply be \textit{experienced} via touch. \\

A \textbf{domain of discourse} is a set of all the things under discussion. It is also called a \textit{universe}, particularly in the field of mathematical logic. It specifies, out of all conceivable objects of study, those objects which are pertinent to the matter at hand. It gives \textit{context} to what we say. \\

A \textit{discourse} is a conversation, and its \textit{domain} is the subject of the conversation. A domain may also refer to an entire field of study, perhaps one whose conversation has been developing for thousands of years (e.g. physics, whose domain is physical reality). Each field of study has an accompanying domain of discourse, the set of objects within its purview. Working in such a field is thus akin to exploring a \textit{universe} populated with such objects. \\

As we observe the objects of a universe, we can make \textit{logical statements} about them. We can also ensure our statements are consistent by using a \textit{formal system of deduction} to prove them from a set of \textit{axioms}. And if we compile every statement that can be proven in this system---the axioms and their consequences---we will form a \textit{theory}, a set of true statements (or \textit{theorems}) that describes the universe in question. \\

The word \textit{theory} may also refer to a \textit{theory-in-progress}, a body of logical work to which theoreticians contribute. We might consider this an \textit{open theory}, a theory for which we do not have all possible knowledge. In contrast, a \textit{closed theory} is one for which we know everything there is to know. \\

Similarly, we can describe universes as open or closed. The \textit{open-world assumption} is made when we do not have complete knowledge of a system: if a statement is not known to be true (i.e. it is not a member of our theory), we do not assume anything about its truth value. This is the case for any system that is \textit{discovered}, like the various systems of nature. \\

Computational systems, however, are not discovered but \textit{designed}. Under normal conditions, it is possible to know everything that happens during a digital computation---all of the information appears in a discrete, finite space of computer memory. Thus, we are omniscient with regard to this universe and should make the \textit{closed-world assumption}: if a statement is not known to be true, it must be false. \\

The \textbf{type-token distinction} \\

\textbf{Identity} seems to be a mandatory quirk of being. It is difficult to imagine an existence in which things are not themselves. And yet, it is also difficult to define, in any meaningful way, this supposedly crucial concept. \\

The \textbf{map-territory relation} \\

% Taxonomy diagram
\begin{center}
    \resizebox{\textwidth}{!}{
        \begin{tikzpicture}[scale=0.2]
         
            %--- GRAY TREES -------------------------
            \foreach \x in {0,...,3}{
                \foreach \y in {1,...,5}{   
                    % Big dashed lines
                    \draw [gray,dashed,thick] (\x*17,0) -- ++(90+\y*15:8);
                    \draw [gray,dashed,thick] (\x*17,0) -- ++(180+\y*15:8);
                     
                    % Small dashed lines
                    \draw [gray,dashed,thick] (\x*17,0) ++(90+\y*15:8) -- ++(90+\y*15-15:3);
                    \draw [gray,dashed,thick] (\x*17,0) ++(90+\y*15:8) -- ++(90+\y*15+15:3);
                    \draw [gray,dashed,thick] (\x*17,0) ++(180+\y*15:8) -- ++(180+\y*15-15:3);
                    \draw [gray,dashed,thick] (\x*17,0) ++(180+\y*15:8) -- ++(180+\y*15+15:3);
                     
                    % Small circles
                    \draw [gray,thick,fill=white] (\x*17,0) ++(90+\y*15:8) circle (0.55);
                    \draw [gray,thick,fill=white] (\x*17,0) ++(180+\y*15:8) circle (0.55);
                }
            }
            % Extra gray tree for Y_1 at 180 degrees
            \draw [gray,dashed,thick] (0,0) -- ++(180:8);
            \draw [gray,dashed,thick] (0,0) ++(180:8) -- ++(180-15:3);
            \draw [gray,dashed,thick] (0,0) ++(180:8) -- ++(180+15:3);
            \draw [gray,thick,fill=white] (0,0) ++(180:8) circle (0.55);
             
            %--- CURVES AND BIG ARROWS --------------
             
            % Curve coordinates
             
            % Red
            \coordinate (R1) at ($(17,0)+(-65:4)$);
            \coordinate (R2) at (24,-12);
            \coordinate (R3) at (20,-22);
             
            % Blue
            \coordinate (M1) at (8,-18);
            \coordinate (M2) at (-4,-14);
            \coordinate (M3) at (5,-1);
             
            % Green
            \coordinate (L1) at (6,-8);
            \coordinate (L2) at (7,-15);
            \coordinate (L3) at (-5.5,-22);
            \coordinate (END) at ($(-12,-10)+(-80:2)$);
             
            % MIDDLE curve
            \draw [thick] plot [smooth]
            (M1) .. controls (M2) and (M3) .. (L1);
             
            % Big arrows
            \draw [thick, rounded corners, fill=white] (-6.5,13) -- ++(0,4) -- ++(50,0) -- ++(0,2) -- ++(9,-4) -- ++(-9,-4) -- ++(0,2) -- cycle;
            \node at (10.5,15) {\Large \textsc{Generalization}};
             
            \draw [thick, rounded corners, fill=white] (54.5,-13) -- ++(0,-4) -- ++(-50,0) -- ++(0,-2) -- ++(-9,4) -- ++(9,4) -- ++(0,-2) -- cycle;
            \node at (37.5,-15) {\Large \textsc{Specification}};
             
            % RIGHT curve
            \draw [thick] plot [smooth]
            (R1) .. controls (R2) and (R3) .. (M1);
             
            % LEFT curve
            \draw [-{Latex[length=5mm,width=2mm,angle=60:6pt]},thick] plot [smooth]
            (L1) .. controls (L2) and (L3) .. (END);
             
            %--- NODES AND SMALL ARROWS -------------
             
            % Arrows
            \foreach \x in {1,2,3}{
                \draw [-{Latex[length=5mm,width=2mm,angle=60:6pt]},thick] (17*\x,0) -- +(-13,0);
            }
            \draw [-{Latex[length=5mm,width=2mm,angle=60:6pt]},thick,dashed] (17*3+8,0) -- +(-4,0);
            
            % Y nodes
            \draw [thick,fill=white] (0,0) circle (4) node {\Large $Y_1$};
            \draw [thick,fill=textbook-blue] (17,0) circle (4) node {\Large $Y_3$};
            \draw [thick,fill=white] (2*17,0) circle (4) node {\Large $Y_2$};
            \draw [thick,fill=white] (3*17,0) circle (4) node {\Large $Y_0$};
                 
            % b node
            \draw [thick,fill=textbook-blue] (-12,-10) circle (2);
            \node at (-12,-10) {\Large $b$};
                 
            % Bezier curve assistant (used for sculpting curves)
            % \draw [thick,red] (R1) -- (R2) -- (R3) -- (M1);
            % \draw [thick,blue] (M1) -- (M2) -- (M3) -- (L1);
            % \draw [thick,green] (L1) -- (L2) -- (L3) -- (END); 
        \end{tikzpicture}
    }
\end{center}

