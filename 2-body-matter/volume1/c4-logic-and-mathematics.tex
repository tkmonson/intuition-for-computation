\chapter{Logic and Mathematics}

\vspace{4mm}
\begin{displayquote}
    \textit{Upon this first, and in one sense this sole, rule of reason, that in order to learn you must desire to learn, and in so desiring not be satisfied with what you already incline to think, there follows one corollary which itself deserves to be inscribed upon every wall of the city of philosophy:
    \vspace{1mm}
    \begin{center} Do not block the way of inquiry. \end{center}}
    \vspace{2mm}
    \begin{flushright}
        ---Charles Sanders Peirce
    \end{flushright}
\end{displayquote}
\vspace{3mm}

\section{The Syntax and Semantics of Argument}

\textit{Reasoning} is a cognitive act performed by rational beings. It involves the absorption and synthesis of presently available information for the purpose of elucidating knowledge. It may require the ingenuity of thought, but in some cases the rote application of simple rules is sufficient. Reasoning could also be described as the providing of good \textit{reasons} to explain or justify things. However, as you might expect, there is much debate over whether or not any particular reason is "good." Reasoning is broad, and it comes in different flavors, but ultimately it is the pursuit of \textit{truth}. \\

Reasoning may involve the application of \textit{logic}. This \textit{logical reasoning} is the kind of reasoning that can be expressed in the form of an \textit{argument}. However, this argument does not need to be "correct." Such reasoning simply must be explainable in \textit{steps}, whether that be through informal speech or formal writing. \\

In logic, an argument has two parts: a set of two or more \textit{premises} and a \textit{conclusion}. If the conclusion \textit{necessarily} follows from the premises, the argument is called \textit{valid}. To know whether or not an argument is valid, one must produce a step-by-step \textit{deduction} of the conclusion from its premises. \\

A deduction is constructed within a \textit{deductive system}, which also has two parts: a set of two or more \textit{premises} and a set of \textit{rules of inference}. A rule of inference can be thought of as a kind of logical function. It takes statements (such as premises) as input and outputs either an intermediate result or the conclusion. A deduction, then, has \textit{three} parts: a set of two or more \textit{premises}, a \textit{conclusion}, and a set of \textit{intermediates}. A single-step deduction is represented below, along with a suitable rule of inference labeled $MP$. Note that the set of intermediates is empty in the case of deductions with only one step. \\[2mm]

% Modus Ponens diagram
\begin{center}
    \begin{tikzpicture}[scale=0.2]
        \tikzstyle{every node}+=[inner sep=0pt]

        \draw [black] (0,0) circle (3);
        \draw (0,0) node {$P$};
        
        \draw [black] (15,0) circle (3);
        \draw (15,0) node {$Q$};
        
        \draw [black] (3,0) -- (12,0);
        \fill [black] (11.2,0.5) -- (11.2,-0.5) -- (12,0);
        
        \draw [black] (0,-10) circle (3);
        \draw (0,-10) node {$P$};
        
        \draw [black, thick] (-9,-17) -- (18,-17);
        
        \draw [black] (0,-24) circle (3);
        \draw (0,-24) node {$Q$};
        
        \draw (-7,0) node {$1^\textit{st}$};
        \draw (-7,-10) node {$2^\textit{nd}$};
        \draw (-7,-24) node {$\textnormal{Con.}$};
        
        %----------------------------------------
        
        \draw [very thick] (26,3) rectangle (48,-27);
        \fill [textbook-blue] (26.1,2.9) rectangle (47.9,-26.9);
        
        \draw (37,-6.5) node {\fontsize{50}{40} $\textnormal{M}$};
        \draw (37,-18.5) node {\fontsize{50}{40} $\textnormal{P}$};
        
        %----------------------------------------
        
        \draw [black, thick] (22,3.07) -- (22,-13.07);
        \draw [black, thick] (22,3) -- (21,3);
        \draw [black, thick] (22,-13) -- (21,-13);
        \draw [black, thick] (22,-5) -- (26,-5);
        \fill [black] (25.2,-4.5) -- (25.2,-5.5) -- (26,-5);
        
        \draw [black, thick] (22,-20.93) -- (22,-27.07);
        \draw [black, thick] (22,-21) -- (21,-21);
        \draw [black, thick] (22,-27) -- (21,-27);
        \draw [black, thick] (22,-24) -- (26,-24);
        \fill [black] (22,-24) -- (22.8,-23.5) -- (22.8,-24.5);
    \end{tikzpicture}
\end{center}
\vspace{5mm}

This is the quintessential rule of inference used in logical reasoning, expressed here in symbols. It is known as \textit{modus ponens} (Latin for "a method that affirms"). The first premise $P\rightarrow Q$ is called a \textit{conditional statement} where $\rightarrow$ is the \textit{implies} operator. It states that if the \textit{antecedent} $P$ were true, it would \textit{imply} that the \textit{consequent} $Q$ would also be true. The second premise $P$ simply states that the statement $P$ is indeed true. Thus, by means of \textit{modus ponens}, we can \textit{infer} from the premises $P\rightarrow Q$ and $P$ that the conclusion $Q$ must be true. \\

Inference is also called \textit{entailment}. Though these terms describe the same logical concept, the latter more aptly describes the relationship between statements that are connected in a deduction. To infer is to come to a conclusion that is not \textit{explicitly} stated in the premises. To entail is to require that something follows. If $P$ \textit{entails} $Q$, this suggests that $Q$ is a \textit{logical consequence} of $P$, which we can express using a \textit{turnstile}: $P\vdash Q$. Inference is an action that humans perform. Entailment is a relationship between logical rules. \\

When the inferences made mirror the entailment, you've got yourself a \textit{valid} argument. However, this does not necessarily mean that your argument is \textit{sound}. \textit{Validity} is a property of arguments with a conclusion that can be \textit{deduced} from its premises according to rules of inference. However, validity says nothing about whether or not the premises have a \textit{meaning} that accurately describes reality. \textit{Soundness}, on the other hand, is a property of valid arguments whose premises are \textit{known truths} or \textit{axioms}. The deduction of a sound argument is called a \textit{proof}. \\

Validity and soundness complicate entailment. Now, an argument can be evaluated either in terms of its \textit{syntax} or in terms of its \textit{semantics}. Syntax refers to the arrangement of symbolic \textit{words} (or, more generally, \textit{tokens}) in a body of text. Semantics, on the other hand, refers to the \textit{meaning} that can be \textit{interpreted} from the text. Both of these concepts are integral to human language and indeed to the communication of information in general. \\

Consider the following valid deduction: \\

\begin{center}
    \begin{tabular*}{0.55\textwidth}{@{\extracolsep{\fill} } ll}
        \textnormal{"If it is raining, it is Tuesday."} & $P\rightarrow Q$ \\
        \textnormal{"It is raining."} & $P$ \\
        \hline
        \textnormal{"Therefore, it is Tuesday."} & $\therefore Q$ \\
    \end{tabular*}
\end{center}
\vspace{4mm}

This conclusion is a \textit{syntactic consequence} of the premises ($P\vdash Q$). That is, when the argument is evaluated simply as a collection of symbolic strings that are manipulated according to rules, $P$ entails $Q$. \\

Syntactic consequence is sometimes expressed with the following statement: "If the premises are true, the conclusion must also be true." However, while this statement does hold for all valid arguments, text evaluated syntactically does not have any notion of truth. It is simply a sequence of tokens. Thus, it may be better to think of valid arguments as "games that are played according to the rules." For example, chess has no inherent meaning associated with it, but it still has rules, and valid games of chess are those that comply with those rules. \\

Now, it might be Tuesday and raining as you are reading this. However, semantically, it is not required to be Tuesday if it is raining. The first premise given above is, in reality, false. However, if we change it to something that we consider true, we can come to a conclusion that is both valid and sound: \\

\begin{center}
    \begin{tabular*}{0.55\textwidth}{@{\extracolsep{\fill} } ll}
        \textnormal{"If it is raining, it is wet outside."} & $P\rightarrow Q$ \\
        \textnormal{"It is raining."} & $P$ \\
        \hline
        \textnormal{"Therefore, it is wet outside."} & $\therefore Q$ \\
    \end{tabular*}
\end{center}
\vspace{4mm}

In this case, the conclusion is a \textit{semantic consequence} of the premises ($P\vDash Q$). In addition to the argument being syntactically logical, it is also semantically true "in all universes." That is, there is no possible scenario where it is raining, and it is not wet outside. This syntactic-semantic difference will come up repeatedly throughout this guide. \\

\section{The Structure of Proof}

Of course, arguments can have more than one step. An argument can involve many inferences, some of which may use \textit{intermediate conclusions} as premises for further conclusions. Multiple threads of reasoning can extend from the premises, weaving through intermediate conclusions and intertwining via inference until they all meet at a \textit{final conclusion}. This is the \textit{tree structure} that is inherent in a \textit{formal proof}. \\

% Tree Structure of Proofs
\begin{center}
    \begin{tikzpicture}[scale=0.2]
        
        \coordinate (L) at (135:10);
        \coordinate (R) at (45:10);
        
        % Level 0
        \draw [black] (0:0) -- ++(L);
        \draw [black] (0:0) -- ++(R);
        
        \fill [white] (0:0) circle (3);
        \draw [black] (0:0) circle (3);
        \draw (0:0) node {$F$};
        
        % Level 1
        \draw [black] (R) -- ++(L);
        \draw [black] (R) -- ++(R);
        
        \fill [white] (L) circle (3);
        \draw [black] (L) circle (3);
        \draw (L) node {$P_1$};
        
        \fill [white] (R) circle (3);
        \draw [black] (R) circle (3);
        \draw (R) node {$I_2$};
        
        % Level 2
        \draw [black] ($ 2*(R) $) -- ++(L);
        \draw [black] ($ 2*(R) $) -- ++(R);
        
        \fill [white] ($ (R) + (L) $) circle (3);
        \draw [black] ($ (R) + (L) $) circle (3);
        \draw ($ (R) + (L) $) node {$P_2$};
        
        \fill [white] ($ 2*(R) $) circle (3);
        \draw [black] ($ 2*(R) $) circle (3);
        \draw ($ 2*(R) $) node {$I_1$};
        
        % Level 3
        \fill [white] ($ 2*(R) + (L) $) circle (3);
        \draw [black] ($ 2*(R) + (L) $) circle (3);
        \draw ($ 2*(R) + (L) $) node {$P_3$};
        
        \fill [white] ($ 3*(R) $) circle (3);
        \draw [black] ($ 3*(R) $) circle (3);
        \draw ($ 3*(R) $) node {$P_4$};
        
        % Arrows        
        \foreach \i in {0,1,2} {
                \coordinate (A) at ($ (45:3) + \i*(R) $);
                \coordinate (B) at ($ (135:3) + \i*(R) $);
                \fill [black,rotate around={45:(A)}] (A) -- ++(0.8,0.5) -- ++(0,-1);
                \fill [black,rotate around={45:(B)}] (B) -- ++(-0.5,0.8) -- ++(1,0);
        }
        
    \end{tikzpicture}
\end{center}
\vspace{4mm}

In this \textit{tree}, the \textit{first premises} are labeled as $P_n$ where $n\in\{1,2,3,4\}$, the \textit{intermediate conclusions} are labeled as $I_m$ where $m\in\{1,2\}$, and the \textit{final conclusion} is labeled $F$. Though one can make a distinction between first premises and intermediate conclusions, it is important to note that they are all premises with regard to the eventual final conclusion. Thus, there are no restrictions on combining them beyond those dictated by the rules of inference. \\

In modern proof theory, proofs like these are studied not only for their semantic meaning, but also for their mathematical structure. They are treated like \textit{data structures}, which are abstract objects that are used in computer science to model the storage and flow of digital \textit{information}. Similarly, a proof is an abstract text that models the flow of logical information as it is manipulated by rules of inference toward a final conclusion. \\

Information, in the context of computer science and, more generally, \textit{information theory}, refers to a \textit{syntactic} message. Formally, it is an order sequence of symbols \\

In computer science, the data structures of interest are \textit{recursive}. That is, the structure can be defined "in terms of itself." \textit{Recursion} is a deep topic, and we will spend much of the next part characterizing it, but a practical way to think about a recursive structure is in terms of \textit{type}. \\

Types are a widespread feature in programming languages. As a concept, they are founded in a logical system called a \textit{type theory}. There are multiple type theories, and some of them are used as alternative foundations for mathematics. We will cover all of this in detail later, but, essentially, a type is a label that can be given to 

However, not all forms of reasoning have such strict rules. There are other methods of reasoning that cannot be modeled by an argument. In contrast to logical reasoning, \textit{intuitive reasoning} has steps that are \textit{not} understood. Although the question might seem peculiar, it is worth asking whether or not computation can be intuitive. So, to begin our journey of understanding computation in a modern, logical sense, we will first walk in the other direction, considering it instead in a mystical, otherworldly sense. \\

\section{Intuitive Reasoning: The Oracle, the Seer, and the Sage}

\textit{Intuition} is the capacity to create conclusions without evidence, proof, or a combination of the two. If any premises are involved, they may appear to an onlooker as if they were plucked out of thin air. If any method is involved, it is esoteric or hidden from sight. Acts of intuition range from the mundane procedures we perform without thinking to great feats of intellectual, artistic, and athletic achievement. And while it is often associated with magic or supernatural ability, intuition is a real, observable phenomenon, and it is a form of reasoning. \\

Like that of reasoning, the definition of intuition is fuzzy. There are a variety of events that one might label as the product of intuition that are actually quite functionally distinct from one another. For this reason, I would like to consider and compare three archetypes that are known for their intuitive skills: the Oracle, the Seer, and the Sage. For the skeptics among you, I ask that you suspend your disbelief for a moment and assume that our characters are acting in good faith. There is no lying here; the conclusions are sincere. \\

\paragraph{The Oracle} \hspace*{1mm} \vspace*{2mm}

An Oracle is a person who predicts the future by acting as a vessel for a god or a set of gods. They are considered by believers to be portals through which the divine speaks directly. They are found in histories all over the world, but most people associate the role with the priestesses of Ancient Greece. Picture a woman with eyes that roll back into her head as she speaks in a possessed, thundering voice. \\

For an Oracle, the conclusions come straight from the source. She may not even do any reasoning herself, save for the \textit{unconscious reasoning} that is performed while she is possessed. However, conscious or not, she provides a great service to mankind. There is no clearer answer than the one given to you directly from the gods themselves, even if it has to be sent through what is essentially the human version of a telephone. \\

For those seeking counsel, the premises of the Oracle's conclusion are unnecessary because they just \textit{know} that the statement must be true. For them, the connection the Oracle has with the gods and with reality is part of life itself. We all have deep beliefs like this. For example, most people do not feel that gravity needs to be proven to them in order for them to accept it as fact. Their perception of gravity in the world around them is proof enough. This is the kind of intuition characterized by \textit{subconscious reasoning}, the reasoning you do without being aware of doing it. \\

\paragraph{The Seer} \hspace*{1mm} \vspace*{2mm}

A Seer is a person who predicts the future by interpreting signs from a god or a set of gods. Unlike an Oracle, a Seer speaks divine truth in his own words, drawing from an innate, sometimes god-given power to see meaning in natural events or occult objects. There are various methods of divination that are used by Seers, such as scrying (the gazing into magical things, like crystal balls, for the purpose of seeing visions), auspicy (the interpretation of bird migration), or dowsing (the use of magic to find water, often with the aid of a dowsing rod). The acceptance of any particular method is cultural, but beliefs in divination vary greatly, even within a single society. \\

The Seer has premises for his conclusion, but the rules of inference involved in his reasoning are incomprehensible to others. He 

\paragraph{The Sage} \hspace*{1mm} \vspace*{2mm}

\begin{tcolorbox}[breakable, enhanced, colback=textbook-blue, sharp corners]
    \vspace{3mm}
    \begin{center}
        \textbf{What is Truth?}
    \end{center}
    Truth, in the absolute sense, has been discussed and debated since the dawn of man. It is a concept that seems obvious to us, and yet it always seems to elude our understanding. Philosophers have formulated dozens of theories of truth over the years, with some asserting that truth is an objective property of our universe and others asserting that truth is a useful lie that mankind has invented. And while there is merit in those claims that truth is not real, we still see around us the technology that was born out of our intuitive sense of true and false. Especially in computer science where everything is represented in binary. \\

    The traditional theories of truth have been termed \textit{substantive}. That is, they assert that truth has some basis in reality and that it is a meaningful thing to discuss. Early theories of truth from Ancient Greece are considered the foundation of \textit{correspondence theory}, the idea that statements are true if their symbols are arranged in such a way that they express an abstract thought that accurately describes reality. This theory defines truth in the context of a relationship between language and objective reality. It is a useful philosophy that allows us to orient ourselves in the world around us. However, many philosophers believe that truth cannot be explained by such a simple rule. In fact, there are many more factors that could play a role in the concept. \\

    Objections to a strict correspondence theory usually take issue with the treatment of language as something monolithic and easy to classify within a true-false dichotomy. For example, a statement encoded as a sentence in a language is only meaningful to people who can read that language. What does this imply about a particular language's relationship with truth? Is a statement only true for those who understand it? Furthermore, there is no guarantee that readers of the same language will even be able to agree on the meaning expressed by a particular sentence. Languages often encompass multiple dialects that might parse words differently. Words themselves also may not be precise, and some abstract thoughts may not have suitable words in some languages. These are the points raised in \textit{social constructivism}, a theory which avers that human knowledge is historically and culturally specific. Social constructivists also believe that truth is \textit{constructed} and that language cannot faithfully represent any kind of objective reality. \\

    Other substantive theories find the essence of truth nestled in other abstract concepts. \textit{Consensus theory}, as the name implies, defines truth as something that is agreed upon either by all human beings or by a subset (such as academic groups or standards bodies). This is another anthropocentric definition that is at constant risk of philosophical division on any given topic. \textit{Coherence theory} takes a more objective approach, claiming that a statement can only be true if it fits into a system of statements that support each other's truth. This is similar to the notion of \textit{formal systems}, which we will discuss in depth later. However, traditional coherence theories attempt to explain all of reality within a single coherent system of truths, which is incompatible with our modern understanding of formal systems. \\

    Modern developments in philosophy have resulted in theories that deviate from the long-held, substantive opinions on the nature of truth. These \textit{minimalist} theories assert instead that truth is either not real or not a necessary property of language or logic. They claim that statements are \textit{assertions} and thus are implicitly understood to be true. For example, it is understood that by putting forth the sentence "$2+2=4$," you are endorsing the semantic meaning "$2+2=4$ is true." The clause "is true" is called a \textit{truth predicate}, and minimalist theories of truth often consider its use redundant. \\

    This idea of a truth predicate is borrowed from Alfred Tarski's \textit{semantic theory of truth}, which is a substantive theory that refines the correspondence concepts espoused by Socrates, Plato, and Aristotle for formal languages. This theory makes a distinction between a statement made in a formal language and a truth predicate, which evaluates the truth of the statement. Tarski made this distinction in order to circumvent the \textit{liar's paradox}, which is often presented with the following example: "This sentence is false." If the predicate "is false" is considered to be part of "this sentence," the truth of this statement cannot be decided. For this reason, Tarski states that a language cannot contain its own truth predicate. This is enforced by requiring that "this sentence" be written in an \textit{object language} and that "is false" be written in a \textit{metalanguage}. \textit{Convention T}, the general form of this rule, can be expressed as

    \begin{center}
        \textit{"$P$" is true if and only if $P$}
    \end{center}
        
    where "$P$" is simply the metalanguage sentence $P$ rendered in the object language. That is, the \textit{syntactic representation} is assigned a \textit{truth value} of true if and only if the semantic meaning it represents is considered true (according to whatever theory of truth you employ). By this rule, we say that "$2+2=4$" is true if and only if \textit{the sum of the number $2$ with the number $2$ is equal to the number $4$}. Minimalist redundancy theories modify Tarski's theory by interpreting "$P$" as an implicit assertion of the truth of $P$. The sentence "$P$," which asserts that $P$ is true, is then false if and only if its semantic meaning $P$ is false. \\

    \parbreak

    The debate on the nature of truth rages on \textit{in perpetuum}. However, for our purposes, we must be practical. There is truth in logic and mathematics and computer science as well, but, in practice, it has little to do with the debates on \textit{absolute truth} described above. Their truth is \textit{relative}. That is, it relies on the assumption that our premises are true. Cognizant of this, we move forward with our thinking, searching for truths within these arbitrary boundaries. \\
    \vspace{3mm}
\end{tcolorbox}
\vspace{2\baselineskip}

\section{Logical Reasoning: The Mathematician, the Scientist, and the Detective}

Logical reasoning is the \textit{inference} of new information from present information. It involves \textit{rule of inferences} that are used to relate sets of \textit{premises} to \textit{conclusions}. There are three kinds of logical reasoning: \textit{deductive}, \textit{inductive}, and \textit{abductive}. Each is classified according to which piece of information (premises, rule, or conclusion) is missing and must be inferred from the others. \\

Inductive arguments exist on a spectrum between \textit{weak} and \textit{strong}. Those that are stronger and more persuasive have a higher probability of having a true conclusion. Inductive reasoning is associated with \textit{science} and \textit{critical thinking} because it allows one to make generalizations about complex phenomena given limited evidence. Unlike deductive reasoning, it attempts to find new knowledge that is not simply contained within its premises. \\

Statements made by induction are bolstered with evidence whereas deductive statements are as true as their premises. This leaves inductive reasoning susceptible to \textit{faulty generalizations} and \textit{biased sampling}. Induction must also assume that future events will occur exactly as they have in the past, which is not always the case. For example, a turkey that is fed every morning with the ring of a bell may infer by induction that bell $\rightarrow$ food. However, he will see the error in his reasoning when the farmer rings the bell on Thanksgiving Day and instead slits his throat. \\

\paragraph{Abductive Reasoning} \hspace*{1mm} \vspace*{2mm}

\textit{Abductive reasoning} is the inference of a \textit{premise}, given a conclusion and a rule. It is investigative in nature. For example, given a conclusion "The grass is wet" and a rule "When it rains, the grass gets wet," we might determine that rain is the best explanation for the wetness of the grass. Thus, we abduce that "It might have rained." \\

Like induction, abduction can also produce a hypothesis. However, abduction does not seek a new relationship between two previously unconnected statements. Rather, it uses established relationships to find a reasonable explanation for a statement that is assumed to be true. It is often used by detectives or diagnosticians who need to find a probable cause of an event. It is also used in Bayesian statistics. While multiple premises may be abduced, typically we want to abduce a single, "best" premise. \\

Abductive reasoning allows us to ignore the many causes that are unlikely in favor of those few that may be relevant to the problem at hand. For example, doctors are often taught to heed the following proverb: "When you hear hoofbeats, think of horses, not zebras." That is, when a patient exhibits certain symptoms, a doctor should abduce from them a commonplace disease before considering more exotic possibilities. However, "zebras" do exist. Sometimes the most likely cause is not the actual cause. For this reason, abduction is also considered to be equivalent to a deductive fallacy called \textit{affirming the consequent}, which is like a \textit{modus ponens} performed in reverse. That is, given a conditional $P\rightarrow Q$ and the consequent $Q$, abduction infers $P$ from $Q$ by assuming that the converse $Q\rightarrow P$ of the conditional is also true. This is not a deductively valid inference. Considering our example again, the grass might be wet from rain, but this is not \textit{necessarily} true. It is also possible that the sprinkler system is on. Or perhaps there is a zebra-esque scenario like a flood. \\\\

