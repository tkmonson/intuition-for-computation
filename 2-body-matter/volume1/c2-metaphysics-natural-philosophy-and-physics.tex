\chapter{Metaphysics, Natural Philosophy, and Physics}

\introquote{I just wondered how things were put together.}
{Claude Shannon}

\section{Classical Metaphysics}

\subsection{Mythology and Cosmogomy}

\subsection{Pre-Socratic Thought: \textit{Henosis} and \textit{Arche}}

\subsection{Democritus' Atomism}

\subsection{Plato's Theory of Forms}

Plato (c. 428/427 -- c. 348/347 BCE) was the first to expound a metaphysics of objects with his \textit{theory of Forms}, which proposed a solution to the \textit{problem of universals}. This problem is subtle, but important: how can one general property appear in many individual objects? For example, when we say that a floor and a bowl are \textit{smooth}, are we referring to some singular paradigm of \textit{smoothness} that is independent of each? For Plato, \textit{universals} or \textit{Forms}, such as smoothness, are indeed distinct from the \textit{particulars} that \textit{partake} of them, such as floors and bowls and other smooth objects. He posits that the \textit{essence} of an object, its \textit{sine qua non}, is the result of its partaking of these Forms.

\textit{Forms}, as conceptualized by Plato, are ideal, abstract entities that exist in the \textit{hyperouranos}, the realm beyond the \textit{heavens}. In the times of Homer and Hesiod, the Ancient Greeks understood a cosmology that closely mirrored that of the Hebrew Bible. They envisioned the Earth as a flat disc surrounded by endless ocean and the sky as a solid dome with gem-like stars embedded in it, much as it all appears from a naive perspective on the ground. Under the Earth was the \textit{underworld}, where the dead live, and beyond the dome were the \textit{heavens}, where the divine live.

Plato, however, was influenced by the philosophy of Pythagoras (c. 570 -- c. 495 BCE), who championed a spherical Earth for aesthetic reasons. Thus, he instead modeled the Earth as a "round body in the center of the heavens," and he imagined the Forms as part of a different, eternal world that transcends physical space and time. Whereas physical reality is the domain of perception and opinion, the realm of the Forms is, as he paints it in \textit{Phaedrus}, the domain of \textit{reason}:

\vspace{2mm}
\begin{displayquote}
    Of that place beyond the heavens none of our earthly poets has yet sung, and none shall sing worthily. But this is the manner of it, for assuredly we must be bold to speak what is true, above all when our discourse is upon truth. It is there that true Being dwells, without colour or shape, that cannot be touched; reason alone, the soul's pilot, can behold it, and all true knowledge is knowledge thereof. Now even as the mind of a god is nourished by reason and knowledge, so also is it with every soul that has a care to receive her proper food; wherefore when at last she has beheld Being she is well content, and contemplating truth she is nourished and prospers, until the heaven's revolution brings her back full circle. And while she is borne round she discerns justice, its very self, and likewise temperance, and knowledge, not the knowledge that is neighbor to Becoming and varies with the various objects to which we commonly ascribe being, but the veritable knowledge of Being that veritably is. And when she has contemplated likewise and feasted upon all else that has true being, she descends again within the heavens and comes back home. And having so come, her charioteer sets his steeds at their manger, and puts ambrosia before them and draught of nectar to drink withal.
\end{displayquote}
\vspace{2mm}

Plato makes a revolutionary statement here about the nature of scientific knowledge. He declares that one cannot know the physical world directly because it is, in fact, \textit{less real} than the \textit{hyperouranos}. As he puts it in \textit{Timaeus}, a Form is "what always is and never becomes," and a particular is "what becomes and never is." Further, physical reality is merely the "receptacle of all Becoming," a three-dimensional \textit{space} that affords spatial \textit{extension} and \textit{location} to all particulars. Unlike Forms, particulars are subject to \textit{change}, and thus cannot be sources of absolute knowledge. They are slippery, metaphysically undefinable, and explainable only in terms of the unchanging Forms of which they partake.

The above passage can be interpreted as an early argument for basing human knowledge in abstract mathematics rather than in subjective evaluations of concrete observations. For example, to \textit{know} what a triangle is, one cannot simply point to a triangular rock or even to a very carefully drawn diagram of a triangle. One must instead use \textit{abstract reasoning} to express the Form of a triangle by means of a \textit{formal definition}: a three-sided shape in a two-dimensional plane. This is the \textit{essence} of a triangle. In contrast, the triangular rock and diagram each partake of multiple Forms, and neither are perfectly triangular. We may still \textit{call} them triangles, but only those who know the Form of a triangle will understand what we mean when we do so. Therefore, it is knowledge of the unambiguous Form that gives insight into the complex and ever-changing particular, not the other way around.

Here, our object-property distinction blurs. Because while Forms may define universal properties, are they not also objects in an abstract sense? And while concrete particulars may be objects, are they not also fully characterized by a set of properties? In Plato's metaphysics, the categories of interest are really the \textit{concrete} and the \textit{abstract}, that which we \textit{perceive} and that which we \textit{conceive}. Plato further claims that humans come from the abstract realm, where they familiarize themselves with the Forms, and they are born into the concrete realm, where particulars are abundant and Forms are only latent memories. It is through philosophy then, the exploration of the rational mind, that we unearth our innate understanding.

To explain the link between the realms, Plato also postulates a \textit{demiurge}, a transcendent artisan who shapes an initially chaotic space into an ordered, material reality, using the Forms as models. This entity is benevolent and wishes to craft a world like the ideal one he sees, but the objects that he \textit{instantiates} are always imperfect copies. Like that of a master portraitist, his work is beautiful, but it is, in the end, a mere \textit{image} of his subject.

So it is for programmers. Before us lies the space of computer memory, uniform and featureless, a receptacle of Becoming. Our Forms are \textit{types}, our particulars are \textit{values}, and the behavior in the space is dictated by the \textit{code} that we execute. In \textit{object-oriented programming} (OOP), values can be of an \textit{object type} that is formally defined by a \textit{class}.

For example, my cat Mystic belongs to the class of entities known as Cat, the set of all possible cats. She is an \textit{object} known as a cat, and her catness is defined by the class Cat. In Plato's terms, she is a \textit{particular}, partaking of the Form \textit{Cat}. Just as the demiurge instantiates particulars from Forms, so the programmer instantiates objects from classes. Both bridge the gap between what is possible and what is.

That said, while Plato's metaphysics gives good insight into OOP and other abstract (or \textit{high-level}) programming concepts, it cannot explain concrete (or \textit{low-level}) ideas like \textit{data} that is represented with \textit{bits}. This is because Plato was not overly concerned with the \textit{substance} of objects, the stuff that makes them \textit{physical}. Thus, in our pursuit of a holistic understanding of \textit{information}, we must also consider \textit{substance theories}, philosophies that give metaphysical weight to \textit{matter}.

\subsection{Aristotle's Hylomorphism}

\section{Theories and Models}

\subsection{Euclid's \textit{Elements}}

Although arithmetic has been practiced since prehistory, theoretical mathematics was founded by the Ancient Greeks. The Greeks, in contrast to earlier peoples, applied \textit{deductive reasoning} to the study of mathematics. While sophisticated arithmetic, geometry, and algebra was done before this period by the Babylonians and the Egyptians, this work was all based on \textit{evidence} collected from the physical world. The Greeks were the first to recognize the need for \textit{proof} of mathematical statements using \textit{logic}.

While various proofs were written by Greeks in the centuries before his birth, Euclid of Alexandria is nonetheless credited with formalizing the concept of proof due to his use of the \textit{axiomatic method} in his groundbreaking treatise, \textit{Elements}. This work was the first of its kind, serving as both a comprehensive compilation of Greek mathematical knowledge and as an introductory text suitable for students. Additionally, it was a product of a synthesis of Greek thought, founded on the epistemology and metaphysics of Plato and Aristotle.

\textit{Elements} is divided into thirteen books. Books I--IV and VI cover \textit{plane geometry}, the study of shape, size, and position in two-dimensional space. Books V and VII--X cover elementary \textit{number theory} (classically known as \textit{arithmetic}) in a geometric way, expressing numbers as lengths of line segments. Finally, Books XI--XIII cover \textit{solid geometry} by applying principles of plane geometry to three-dimensional space. Each book begins with a list of \textit{definitions} (if necessary) that assign names to abstract mathematical ideas that are expressed in unambiguous terms. These ideas are then used to prove \textit{propositions}, mathematical statements that warrant proof.

Euclid's proofs in \textit{Elements} can be described as \textit{axiomatic}. That is, they start from \textit{axioms}, statements that are taken as true without proof. Axioms solve a problem in logical argumentation known as \textit{infinite regress}, which often occurs, for example, when children ask questions. A curious child, seeking knowledge, might ask his parent why something is the way it is. That is, he requests a proof of some proposition $P_0$. The parent would respond with some proposition $P_1$ that supports the truth of $P_0$. The child, unsatisfied, would subsequently ask why $P_1$ is true. The parent would respond with the assertion that $P_2$ is true and implies $P_1$, and the child would similarly question whence $P_2$ arises. This dialogue could continue in this manner forever with the child endlessly searching for an anchor with which he can ground his understanding. However, patience waning, the parent must end this line of questioning by giving an unquestionable answer. Common examples include "Because that's just the way it is" or "Because I said so."

Some Greek philosophers considered infinite regress to be a serious epistemological issue. If we assume that all knowledge must be demonstrable by a proof or logical argument, we cannot know anything because our axioms cannot be satisfactorily proven. And yet, there are things that we claim to know. In his \textit{Posterior Analytics}, Aristotle challenges this notion that all knowledge must be provable, averring that

\vspace{2mm}
\begin{displayquote}
    All instruction given or received by way of argument proceeds from pre-existent knowledge.
\end{displayquote} 
\vspace{2mm}

Aristotle continues his text in support of this statement. He discusses the nature of \textit{premises} in his \textit{syllogistic logic} and how they differ from the \textit{conclusions} that they derive. Namely, he states that the premises of a knowledge-producing deductive argument or \textit{syllogism} must be \textit{true}, to ensure the truth of their conclusion, \textit{indemonstrable}, to ensure that they are independent of proof, and \textit{causal}, to ensure that their conclusion logically follows:

\vspace{2mm}
\begin{displayquote}
    Assuming then that my thesis as to the nature of scientific knowledge is correct, the premises of demonstrated knowledge must be true, primary, immediate, better known than and prior to the conclusion, which is further related to them as effect to cause. Unless these conditions are satisfied, the basic truths will not be 'appropriate' to the conclusion. Syllogism there may indeed be without these conditions, but such syllogism, not being productive of scientific knowledge, will not be demonstration. The premises must be true: for that which is non-existent cannot be known---we cannot know, e.g. that the diagonal of a square is commensurate with its side. The premises must be primary and indemonstrable; otherwise they will require demonstration in order to be known, since to have knowledge, if it be not accidental knowledge, of things which are demonstrable, means precisely to have a demonstration of them. The premises must be causes, since we posess scientific knowledge of a thing only when we know its cause; prior, in order to be causes; antecedently known, this antecedent knowledge being not our mere understanding of the meaning, but knowledge of the fact as well.
\end{displayquote}
\vspace{2mm}

Thus, Aristotle differentiates \textit{pre-existent knowledge} from \textit{scientific knowledge}. Whereas the latter is produced by means of a logical proof, the former is produced by one's \textit{intuition}. He goes on to argue that what we intuitively understand is indeed knowledge, in the same way that what we prove is knowledge. The difference, he argues, is that axioms are \textit{self-evident}. For example, the reflexive property of the natural numbers (i.e. for every natural $x$, $x=x$) is typically considered so obvious that proof is unnecessary. At the same time, the statement is so fundamental that it cannot really be proven. Despite their unprovability, Aristotle argues that axioms such as these constitute knowledge that is simply, unquestionably \textit{known}:

\vspace{2mm}
\begin{displayquote}
    Our own doctrine is that not all knowledge is demonstrative: on the contrary, knowledge of the immediate premises is independent of demonstration. (The necessity of this is obvious; for since we must know the prior premises from which the demonstration is drawn, and since the regress must end in immediate truths, those truths must be indemonstrable.) Such, then, is our doctrine, and in addition we maintain that besides scientific knowledge there is its originative source which enables us to recognize the definitions.
\end{displayquote}
\vspace{2mm}

Euclid apparently agreed with this epistemology because Book I of \textit{Elements}, in addition to its definitions and propositions, contains a number of axioms, which are labeled either as \textit{postulates} or \textit{common notions}. In antiquity, a postulate denoted an axiom of a particular science, whereas a common notion denoted an axiom common to all sciences. Today, we do not think of axioms in this way. Rather, we simply consider them starting points for proofs. Euclid's axioms are given below:

\begin{center}
    \textbf{Postulates}
    \begin{enumerate}
        \item Hi
    \end{enumerate}
    \textbf{Common Notions}
    \begin{enumerate}
        \item Hi
    \end{enumerate}  
\end{center}

Euclid's proofs can also be described as \textit{constructive}.

\section{Ancient Knowledge, Lost and Found}

\section{Particles and Forces}

\section{Waves and Fields}

\section{Old Quantum Theory and Relativity}

\section{Quantum States and Operators}

\section{Information Theory}

