\chapter{Preface}

% Use a typewriter font for the whole preface?

% Search for section references at the end of the writing process and use latex commands for them.

\vspace{4mm}
\begin{displayquote}
    \textit{I wrote my first novel because I wanted to read it.}
    \begin{flushright}
        ---Toni Morrison
    \end{flushright}
    \vspace{4mm}
\end{displayquote}

% I just read it. It's good. And coming from someone who is so critical of himself, that's high praise.

% My life has been very strange for the past several months.
% Many believe that thought processes become rigid somewhere between one's mid-20s and early 30s. This notion has become more and more concerning to me as I have inched toward that threshold.

% Start with hearing "A Crow Looked At Me" in 2017.
% "Quasi-graduating" in mechanical engineering and coming back for an extra semester to snag a major in computer science as well.

% This process has made my mind feel like a small planet, ripped in two and tossed into space, forcefully opened. Just cracked like a coconut into its halves.
% I have become a different person because of writing this book, perhaps not outwardly but inwardly. And that is what really matters. That is the guy I have to live with.
% I have found the root of my malaise, and I have burned it to ash.

% In December of 2018, I graduated from Duke University, feeling like I knew simultaneously very much and nothing at all.

% Why did you write this?
% Why do you think its worth sharing with others?
% Why read this over a traditional textbook?
% Introduce the book as a path or journey?

% College gave me reasoning and knowledge. Writing this guide gave me intuition.
% When you get to the end of any extended course of study, you tend to gain *insights*.

% *** I don't understand why these insights need to be so hard-earned. Why can't there just be a textbook full of the insights? ***

% Categorizing things is an age old approach to philosophy and science. It is also a technique that plays a fundamental role in all of human cognition.
% Wax poetic about computer science.
% At times, I struggled writing this book. But I eventually came to a point where I knew what this book was trying to be, and I could envision it. I could see the pieces linking together, and I wanted it to exist. I wanted to lay out, in simple English, the connections I was making in my mind. And I wanted to share these difficult concepts in a way that was accessible.
% This project went from a means to an end to an end in itself.
% Talk about what it was like to just live in this document for such a long time. What is it like to live primarily virtually?
% The value in spending time just philosophizing about things.
% Belief that the most significant, cutting-edge, and interesting jobs in many fields require or would benefit greatly from a deep philosophical and mathematical understanding of software.
% In the end, I was tired of being confused.
% At a certain point, I stopped worrying about what was important to know for technical interviews. I was making educated decisions daily about what was essential and what was optional, what was foundational and what was derivative.
% Mechanical engineering vs. computer science. They have many similarities, and my true interests likely lie between the two.
% Sometimes writing this was a Herculean effort. Sometimes it wrote itself, and I just sat there and watched it happen.
% The physical and mental toll of writing this book.
% Was it worth it?---all the time and effort and isolation? I feel I have given myself a great gift. And now I would like to offer this gift to you.
% I truly feel that I am now capable of anything because I have the right mindest. That is the most important thing that a man or woman can build for him- or herself. Your appraisal of everything you experience depends on your mindset.
% In investigating the content below, I feel that I have discovered a beautiful new way of thinking about the world, and it makes me---an on-again-off-again-now-hopefully-off crippling depressive and abandonless self-defamer---confident that I will do great things in life.
% Because this project was so hard and because I felt the pressure of time, I was forced to streamline my habits, "For the Good of the Book" (TM). I have washed away the cruft from my everyday life, and I feel, with great intensity, that I am aware of what matters and what is worth my time.

% When I started this document, it had sections for job interview related things in addition to the first section which was called something like "Theoretical Knowledge." I quickly realized that that section would be much larger than the rest, so I decided to keep my list theoretical.
% My guiding principle from the start was a deep feeling that data structures were mathematical objects. I thought that there was a direct link between the two and that it could be easily understood if it were explained well.
% I went into the project with a lot of prior beliefs, and at first I was content with giving only general notions of concepts that I thought were especially abstract.
% Because of this lax attitude, I often found myself running into contradictions or becoming paralyzed, unable to write anything with confidence.
% Eventually, I realized that I was fooling myself into thinking that I understood what I was talking about. I began to do more serious research, even on topics that I thought I understood.
% I discovered that real learning can't be done by reading casually through a single textbook. It requires the consumption of content from many authors, in many formats, with much time allowed for one's own imagination and critical thought.
% I treated myself poorly and subjected myself to insane work hours. I very slowly realized that I was only hindering my progress by being so harsh to myself.
% I began productively and then seized up, producing very little. I struggled to find the balance between poetic and informative, the conceptual order of the narrative, and passion without consequent obsession.

% Use that levels of computer science thing: electrical physics, digital logic, microarchitecture, ISA, OS, assembly, high-level languages. This book sort of follows that order (i.e. bottom-up).

% I seem to have a long-standing affinity for metaphors about longing for something just outside of your grasp.
% Quote the ending lyrics from Soria Moria. "I'm an arrow now. Mid air."
% Seeing somewhere that you want to be, but it is really far away.
% Other similar examples: green light from "The Great Gatsby," the paper boat quote from "Dear Esther," Haken's "The Mountain," a lot of my favorite albums, the scene from "The Things They Carried" at the US-Canadian border.

% Sign just with initials T.M.?

